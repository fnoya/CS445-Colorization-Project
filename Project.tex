% Options for packages loaded elsewhere
\PassOptionsToPackage{unicode}{hyperref}
\PassOptionsToPackage{hyphens}{url}
%
\documentclass[
]{article}
\usepackage{booktabs}
\usepackage{amsmath,amssymb}
\usepackage{lmodern}
\usepackage{iftex}
\ifPDFTeX
  \usepackage[T1]{fontenc}
  \usepackage[utf8]{inputenc}
  \usepackage{textcomp} % provide euro and other symbols
\else % if luatex or xetex
  \usepackage{unicode-math}
  \defaultfontfeatures{Scale=MatchLowercase}
  \defaultfontfeatures[\rmfamily]{Ligatures=TeX,Scale=1}
\fi
% Use upquote if available, for straight quotes in verbatim environments
\IfFileExists{upquote.sty}{\usepackage{upquote}}{}
\IfFileExists{microtype.sty}{% use microtype if available
  \usepackage[]{microtype}
  \UseMicrotypeSet[protrusion]{basicmath} % disable protrusion for tt fonts
}{}
\makeatletter
\@ifundefined{KOMAClassName}{% if non-KOMA class
  \IfFileExists{parskip.sty}{%
    \usepackage{parskip}
  }{% else
    \setlength{\parindent}{0pt}
    \setlength{\parskip}{6pt plus 2pt minus 1pt}}
}{% if KOMA class
  \KOMAoptions{parskip=half}}
\makeatother
\usepackage{xcolor}
\IfFileExists{xurl.sty}{\usepackage{xurl}}{} % add URL line breaks if available
\IfFileExists{bookmark.sty}{\usepackage{bookmark}}{\usepackage{hyperref}}
\hypersetup{
  pdftitle={Selective Coloring},
  pdfauthor={Francisco Noya (fnoya2@illinois.edu); Mesay Taye (mesayst2@illinois.edu)},
  pdfkeywords={Colorization, Segmentation},
  hidelinks,
  pdfcreator={LaTeX via pandoc}}
\urlstyle{same} % disable monospaced font for URLs
\usepackage{color}
\usepackage{fancyvrb}
\newcommand{\VerbBar}{|}
\newcommand{\VERB}{\Verb[commandchars=\\\{\}]}
\usepackage{fvextra}
\DefineVerbatimEnvironment{Highlighting}{Verbatim}{breaklines,commandchars=\\\{\}}
% Add ',fontsize=\small' for more characters per line
\usepackage{framed}
\definecolor{shadecolor}{RGB}{248,248,248}
\newenvironment{Shaded}{\begin{snugshade}}{\end{snugshade}}
\newcommand{\AlertTok}[1]{\textcolor[rgb]{0.94,0.16,0.16}{#1}}
\newcommand{\AnnotationTok}[1]{\textcolor[rgb]{0.56,0.35,0.01}{\textbf{\textit{#1}}}}
\newcommand{\AttributeTok}[1]{\textcolor[rgb]{0.77,0.63,0.00}{#1}}
\newcommand{\BaseNTok}[1]{\textcolor[rgb]{0.00,0.00,0.81}{#1}}
\newcommand{\BuiltInTok}[1]{#1}
\newcommand{\CharTok}[1]{\textcolor[rgb]{0.31,0.60,0.02}{#1}}
\newcommand{\CommentTok}[1]{\textcolor[rgb]{0.56,0.35,0.01}{\textit{#1}}}
\newcommand{\CommentVarTok}[1]{\textcolor[rgb]{0.56,0.35,0.01}{\textbf{\textit{#1}}}}
\newcommand{\ConstantTok}[1]{\textcolor[rgb]{0.00,0.00,0.00}{#1}}
\newcommand{\ControlFlowTok}[1]{\textcolor[rgb]{0.13,0.29,0.53}{\textbf{#1}}}
\newcommand{\DataTypeTok}[1]{\textcolor[rgb]{0.13,0.29,0.53}{#1}}
\newcommand{\DecValTok}[1]{\textcolor[rgb]{0.00,0.00,0.81}{#1}}
\newcommand{\DocumentationTok}[1]{\textcolor[rgb]{0.56,0.35,0.01}{\textbf{\textit{#1}}}}
\newcommand{\ErrorTok}[1]{\textcolor[rgb]{0.64,0.00,0.00}{\textbf{#1}}}
\newcommand{\ExtensionTok}[1]{#1}
\newcommand{\FloatTok}[1]{\textcolor[rgb]{0.00,0.00,0.81}{#1}}
\newcommand{\FunctionTok}[1]{\textcolor[rgb]{0.00,0.00,0.00}{#1}}
\newcommand{\ImportTok}[1]{#1}
\newcommand{\InformationTok}[1]{\textcolor[rgb]{0.56,0.35,0.01}{\textbf{\textit{#1}}}}
\newcommand{\KeywordTok}[1]{\textcolor[rgb]{0.13,0.29,0.53}{\textbf{#1}}}
\newcommand{\NormalTok}[1]{#1}
\newcommand{\OperatorTok}[1]{\textcolor[rgb]{0.81,0.36,0.00}{\textbf{#1}}}
\newcommand{\OtherTok}[1]{\textcolor[rgb]{0.56,0.35,0.01}{#1}}
\newcommand{\PreprocessorTok}[1]{\textcolor[rgb]{0.56,0.35,0.01}{\textit{#1}}}
\newcommand{\RegionMarkerTok}[1]{#1}
\newcommand{\SpecialCharTok}[1]{\textcolor[rgb]{0.00,0.00,0.00}{#1}}
\newcommand{\SpecialStringTok}[1]{\textcolor[rgb]{0.31,0.60,0.02}{#1}}
\newcommand{\StringTok}[1]{\textcolor[rgb]{0.31,0.60,0.02}{#1}}
\newcommand{\VariableTok}[1]{\textcolor[rgb]{0.00,0.00,0.00}{#1}}
\newcommand{\VerbatimStringTok}[1]{\textcolor[rgb]{0.31,0.60,0.02}{#1}}
\newcommand{\WarningTok}[1]{\textcolor[rgb]{0.56,0.35,0.01}{\textbf{\textit{#1}}}}
\usepackage{graphicx}
\makeatletter
\def\maxwidth{\ifdim\Gin@nat@width>\linewidth\linewidth\else\Gin@nat@width\fi}
\def\maxheight{\ifdim\Gin@nat@height>\textheight\textheight\else\Gin@nat@height\fi}
\makeatother
% Scale images if necessary, so that they will not overflow the page
% margins by default, and it is still possible to overwrite the defaults
% using explicit options in \includegraphics[width, height, ...]{}
\setkeys{Gin}{width=\maxwidth,height=\maxheight,keepaspectratio}
% Set default figure placement to htbp
\makeatletter
\def\fps@figure{htbp}
\makeatother
\setlength{\emergencystretch}{3em} % prevent overfull lines
\providecommand{\tightlist}{%
  \setlength{\itemsep}{0pt}\setlength{\parskip}{0pt}}
\setcounter{secnumdepth}{-\maxdimen} % remove section numbering
\newlength{\cslhangindent}
\setlength{\cslhangindent}{1.5em}
\newlength{\csllabelwidth}
\setlength{\csllabelwidth}{3em}
\newlength{\cslentryspacingunit} % times entry-spacing
\setlength{\cslentryspacingunit}{\parskip}
\newenvironment{CSLReferences}[2] % #1 hanging-ident, #2 entry spacing
 {% don't indent paragraphs
  \setlength{\parindent}{0pt}
  % turn on hanging indent if param 1 is 1
  \ifodd #1
  \let\oldpar\par
  \def\par{\hangindent=\cslhangindent\oldpar}
  \fi
  % set entry spacing
  \setlength{\parskip}{#2\cslentryspacingunit}
 }%
 {}
\usepackage{calc}
\newcommand{\CSLBlock}[1]{#1\hfill\break}
\newcommand{\CSLLeftMargin}[1]{\parbox[t]{\csllabelwidth}{#1}}
\newcommand{\CSLRightInline}[1]{\parbox[t]{\linewidth - \csllabelwidth}{#1}\break}
\newcommand{\CSLIndent}[1]{\hspace{\cslhangindent}#1}
\ifLuaTeX
  \usepackage{selnolig}  % disable illegal ligatures
\fi

\title{Selective Coloring}
\author{Francisco Noya (fnoya2@illinois.edu) \and Mesay Taye
(mesayst2@illinois.edu)}
\date{2022-05-01}

\begin{document}
\maketitle

In this project we implemented a tool to selectively and automatically
colorize the foreground of black and white photographs. The tool
consists of two main algorithms. First, a GAN network colorizes the
entire photograph using transfer learning. Second, a segmentation
network segments the foreground from the background. Lastly, it blends
the colorized foreground with the original background to produce a
distinctive image.

In this notebook the Colorization network is trained and validated.

\hypertarget{colorization-network}{%
\subsection{Colorization Network}\label{colorization-network}}

\hypertarget{approach}{%
\subsubsection{Approach}\label{approach}}

For the colorization task we worked on the \textbf{Lab} colorspace. In
this space, the \textbf{L} channel contains the luminance or intensity
information, while the \textbf{ab} channels contain the color
information. Therefore, a neural network can be trained with the
\textbf{L} channel of regular color images as input. Its predictions
will be ``fake'' \textbf{ab} channels and its loss will be calculated
with the ``real'' \textbf{ab} channels.

After a short bibliographic review we found that although traditional
convolutional neural networks (CNNs) could produce results almost
indistinguishable from real color photos (Zhang et al., 2016),
\emph{Generative Adversarial Networks} or GANs (Goodfellow et al., 2014)
were the most proper approach for this kind of problem. This network
architecture contains two modules, a Generator and a Discriminator. Both
models are trained in paralell. The objective of the generator is to
produce outputs similar enough to the ground truth that can fool the
discriminator. The discriminator's objective is to properly tell the
ground truth from the discriminator output.

Since training this kind of networks requires large datasets and
computing time, we decided to use pretrained models that have been used
for other tasks such as object classification. We followed a tutorial
inspired by the \emph{pix2pix} paper (Isola et al., 2016) but instead of
training a naïve \emph{UNet} as the generator, it used a \emph{ResNet18}
network as the generator (\emph{Colorizing Black \& White Images with
u-Net and Conditional GAN --- a Tutorial}, n.d.). Similar to
\emph{pix2pix} we used a patch discriminator that splits the image in 26
square patches (depending on the image size) and produces a \emph{real}
or \emph{fake} prediction for each of them.

In order to try different approaches, we decided to use
\emph{Transformers} in place of the discriminator and the generator.
\emph{Transformers} are a special architectures of DNN that make
extensive use of attention mechanisms (Vaswani et al., 2017). Because of
their ability to have larger receptive fields compared to convolutional
neural networks (CNNs) that allow tracking long-range dependencies
within an image, these attention based architectures have proven very
effective in image processing tasks and gave rise to Visual Transformers
or \textbf{ViT} (Dosovitskiy et al., 2020). We tried different
architectures with ViTs generators or discriminators and measured a
range of metrics for each of them.

\begin{Shaded}
\begin{Highlighting}[]
\OperatorTok{!}\NormalTok{pip install fastai }
\end{Highlighting}
\end{Shaded}

\begin{Shaded}
\begin{Highlighting}[]
\ImportTok{import}\NormalTok{ os}
\ImportTok{import}\NormalTok{ glob}
\ImportTok{import}\NormalTok{ time}
\ImportTok{import}\NormalTok{ numpy }\ImportTok{as}\NormalTok{ np}
\ImportTok{from}\NormalTok{ PIL }\ImportTok{import}\NormalTok{ Image}
\ImportTok{from}\NormalTok{ pathlib }\ImportTok{import}\NormalTok{ Path}
\ImportTok{from}\NormalTok{ tqdm.notebook }\ImportTok{import}\NormalTok{ tqdm}
\ImportTok{import}\NormalTok{ matplotlib.pyplot }\ImportTok{as}\NormalTok{ plt}
\ImportTok{from}\NormalTok{ skimage.color }\ImportTok{import}\NormalTok{ rgb2lab, lab2rgb}

\ImportTok{import}\NormalTok{ torch}
\ImportTok{import}\NormalTok{ cv2}
\ImportTok{from}\NormalTok{ torch }\ImportTok{import}\NormalTok{ nn, optim}
\ImportTok{from}\NormalTok{ torchvision }\ImportTok{import}\NormalTok{ transforms}
\ImportTok{from}\NormalTok{ torchvision.utils }\ImportTok{import}\NormalTok{ make\_grid}
\ImportTok{from}\NormalTok{ torch.utils.data }\ImportTok{import}\NormalTok{ Dataset, DataLoader}

\ImportTok{from}\NormalTok{ fastai.vision.learner }\ImportTok{import}\NormalTok{ create\_body}
\ImportTok{from}\NormalTok{ torchvision.models.resnet }\ImportTok{import}\NormalTok{ resnet18}
\ImportTok{from}\NormalTok{ torchvision.models }\ImportTok{import}\NormalTok{ vit\_b\_16 }\ImportTok{as}\NormalTok{ visiontransformer}
\ImportTok{from}\NormalTok{ fastai.vision.models.unet }\ImportTok{import}\NormalTok{ DynamicUnet}
\ImportTok{from}\NormalTok{ fastai.data.external }\ImportTok{import}\NormalTok{ untar\_data, URLs}

\NormalTok{device }\OperatorTok{=}\NormalTok{ torch.device(}\StringTok{"cuda"} \ControlFlowTok{if}\NormalTok{ torch.cuda.is\_available() }\ControlFlowTok{else} \StringTok{"cpu"}\NormalTok{)}
\BuiltInTok{print}\NormalTok{ (device)}
\end{Highlighting}
\end{Shaded}

\begin{verbatim}
cpu
\end{verbatim}

\hypertarget{dataset}{%
\subsubsection{Dataset}\label{dataset}}

For training and validation purposes we used a subset of the COCO
dataset of images (Lin et al., 2014) that is provided by the FastAI
framework (\emph{Fast.ai}, n.d.). We downloaded 10.000 images from this
dataset and randomly splited them into two sets: a training set with
8.000 images and a validation set with 2.000 images.

\begin{Shaded}
\begin{Highlighting}[]
\NormalTok{ coco\_path }\OperatorTok{=}\NormalTok{ untar\_data(URLs.COCO\_SAMPLE)}
\NormalTok{ coco\_path }\OperatorTok{=} \BuiltInTok{str}\NormalTok{(coco\_path) }\OperatorTok{+} \StringTok{"/train\_sample"}
\end{Highlighting}
\end{Shaded}

\begin{Shaded}
\begin{Highlighting}[]
\NormalTok{path }\OperatorTok{=}\NormalTok{ coco\_path}
    
\NormalTok{paths }\OperatorTok{=}\NormalTok{ glob.glob(path }\OperatorTok{+} \StringTok{"/*.jpg"}\NormalTok{) }
\NormalTok{np.random.seed(}\DecValTok{123}\NormalTok{)}
\NormalTok{paths\_subset }\OperatorTok{=}\NormalTok{ np.random.choice(paths, }\DecValTok{10\_000}\NormalTok{, replace}\OperatorTok{=}\VariableTok{False}\NormalTok{) }\CommentTok{\# choosing 10000 images randomly}
\NormalTok{rand\_idxs }\OperatorTok{=}\NormalTok{ np.random.permutation(}\DecValTok{10\_000}\NormalTok{)}
\NormalTok{train\_idxs }\OperatorTok{=}\NormalTok{ rand\_idxs[:}\DecValTok{8000}\NormalTok{] }\CommentTok{\# choosing the first 8000 as training set}
\NormalTok{val\_idxs }\OperatorTok{=}\NormalTok{ rand\_idxs[}\DecValTok{8000}\NormalTok{:] }\CommentTok{\# choosing last 2000 as validation set}
\NormalTok{train\_paths }\OperatorTok{=}\NormalTok{ paths\_subset[train\_idxs]}
\NormalTok{val\_paths }\OperatorTok{=}\NormalTok{ paths\_subset[val\_idxs]}
\BuiltInTok{print}\NormalTok{(}\BuiltInTok{len}\NormalTok{(train\_paths), }\BuiltInTok{len}\NormalTok{(val\_paths))}

\NormalTok{fig, axes }\OperatorTok{=}\NormalTok{ plt.subplots(}\DecValTok{4}\NormalTok{, }\DecValTok{4}\NormalTok{, figsize}\OperatorTok{=}\NormalTok{(}\DecValTok{10}\NormalTok{, }\DecValTok{10}\NormalTok{))}
\NormalTok{fig.suptitle(}\StringTok{\textquotesingle{}Sample images from COCO dataset\textquotesingle{}}\NormalTok{)}
\ControlFlowTok{for}\NormalTok{ ax, img\_path }\KeywordTok{in} \BuiltInTok{zip}\NormalTok{(axes.flatten(), train\_paths):}
\NormalTok{    ax.imshow(Image.}\BuiltInTok{open}\NormalTok{(img\_path))}
\NormalTok{    ax.axis(}\StringTok{"off"}\NormalTok{)}
\end{Highlighting}
\end{Shaded}

\begin{verbatim}
8000 2000
\end{verbatim}

\begin{figure}
\centering
\includegraphics{Project_files/Project_8_1.png}
\caption{png}
\end{figure}

Then we resized the images so that they have manageable dimensions that
allow feeding into the different network architectures without requiring
extremely high computational resources or long times. Similar to (Isola
et al., 2016) data augmentation was achieved by flipping the images
horizontally (this is only done for the training set). We used 16 images
on each batch that goes through the network. Each image was converted to
the \textbf{Lab} colorspace and the channels adjusted float values
between -1 and 1.

\begin{Shaded}
\begin{Highlighting}[]
\NormalTok{SIZE }\OperatorTok{=} \DecValTok{224}
\KeywordTok{class}\NormalTok{ ColorizationDataset(Dataset):}
    \KeywordTok{def} \FunctionTok{\_\_init\_\_}\NormalTok{(}\VariableTok{self}\NormalTok{, paths, split}\OperatorTok{=}\StringTok{\textquotesingle{}train\textquotesingle{}}\NormalTok{):}
        \ControlFlowTok{if}\NormalTok{ split }\OperatorTok{==} \StringTok{\textquotesingle{}train\textquotesingle{}}\NormalTok{:}
            \VariableTok{self}\NormalTok{.transforms }\OperatorTok{=}\NormalTok{ transforms.Compose([}
\NormalTok{                transforms.Resize((SIZE, SIZE),  Image.Resampling.BICUBIC),}
\NormalTok{                transforms.RandomHorizontalFlip(), }\CommentTok{\# A little data augmentation!}
\NormalTok{            ])}
        \ControlFlowTok{elif}\NormalTok{ split }\OperatorTok{==} \StringTok{\textquotesingle{}val\textquotesingle{}}\NormalTok{:}
            \VariableTok{self}\NormalTok{.transforms }\OperatorTok{=}\NormalTok{ transforms.Resize((SIZE, SIZE),  Image.Resampling.BICUBIC)}
        
        \VariableTok{self}\NormalTok{.split }\OperatorTok{=}\NormalTok{ split}
        \VariableTok{self}\NormalTok{.size }\OperatorTok{=}\NormalTok{ SIZE}
        \VariableTok{self}\NormalTok{.paths }\OperatorTok{=}\NormalTok{ paths}
    
    \KeywordTok{def} \FunctionTok{\_\_getitem\_\_}\NormalTok{(}\VariableTok{self}\NormalTok{, idx):}
\NormalTok{        img }\OperatorTok{=}\NormalTok{ Image.}\BuiltInTok{open}\NormalTok{(}\VariableTok{self}\NormalTok{.paths[idx]).convert(}\StringTok{"RGB"}\NormalTok{)}
\NormalTok{        img }\OperatorTok{=} \VariableTok{self}\NormalTok{.transforms(img)}
\NormalTok{        img }\OperatorTok{=}\NormalTok{ np.array(img)}
\NormalTok{        img\_lab }\OperatorTok{=}\NormalTok{ rgb2lab(img).astype(}\StringTok{"float32"}\NormalTok{) }\CommentTok{\# Converting RGB to L*a*b}
\NormalTok{        img\_lab }\OperatorTok{=}\NormalTok{ transforms.ToTensor()(img\_lab)}
\NormalTok{        L }\OperatorTok{=}\NormalTok{ img\_lab[[}\DecValTok{0}\NormalTok{], ...] }\OperatorTok{/} \FloatTok{50.} \OperatorTok{{-}} \FloatTok{1.} \CommentTok{\# Between {-}1 and 1}
\NormalTok{        ab }\OperatorTok{=}\NormalTok{ img\_lab[[}\DecValTok{1}\NormalTok{, }\DecValTok{2}\NormalTok{], ...] }\OperatorTok{/} \FloatTok{110.} \CommentTok{\# Between {-}1 and 1}
        
        \ControlFlowTok{return}\NormalTok{ \{}\StringTok{\textquotesingle{}L\textquotesingle{}}\NormalTok{: L, }\StringTok{\textquotesingle{}ab\textquotesingle{}}\NormalTok{: ab\}}
    
    \KeywordTok{def} \FunctionTok{\_\_len\_\_}\NormalTok{(}\VariableTok{self}\NormalTok{):}
        \ControlFlowTok{return} \BuiltInTok{len}\NormalTok{(}\VariableTok{self}\NormalTok{.paths)}

\KeywordTok{def}\NormalTok{ make\_dataloaders(batch\_size}\OperatorTok{=}\DecValTok{16}\NormalTok{, n\_workers}\OperatorTok{=}\DecValTok{2}\NormalTok{, pin\_memory}\OperatorTok{=}\VariableTok{True}\NormalTok{, }\OperatorTok{**}\NormalTok{kwargs): }\CommentTok{\# A handy function to make our dataloaders}
    \ControlFlowTok{if}\NormalTok{ (os.name}\OperatorTok{==}\StringTok{"nt"}\NormalTok{):}
\NormalTok{        n\_workers }\OperatorTok{=} \DecValTok{0}  \CommentTok{\# In Windows, n\_workers should be 0}
\NormalTok{    dataset }\OperatorTok{=}\NormalTok{ ColorizationDataset(}\OperatorTok{**}\NormalTok{kwargs)}
\NormalTok{    dataloader }\OperatorTok{=}\NormalTok{ DataLoader(dataset, batch\_size}\OperatorTok{=}\NormalTok{batch\_size, num\_workers}\OperatorTok{=}\NormalTok{n\_workers,}
\NormalTok{                            pin\_memory}\OperatorTok{=}\NormalTok{pin\_memory)}
    \ControlFlowTok{return}\NormalTok{ dataloader}
\end{Highlighting}
\end{Shaded}

\begin{Shaded}
\begin{Highlighting}[]
\NormalTok{train\_dl }\OperatorTok{=}\NormalTok{ make\_dataloaders(paths}\OperatorTok{=}\NormalTok{train\_paths, split}\OperatorTok{=}\StringTok{\textquotesingle{}train\textquotesingle{}}\NormalTok{)}
\NormalTok{val\_dl }\OperatorTok{=}\NormalTok{ make\_dataloaders(paths}\OperatorTok{=}\NormalTok{val\_paths, split}\OperatorTok{=}\StringTok{\textquotesingle{}val\textquotesingle{}}\NormalTok{)}

\NormalTok{data }\OperatorTok{=} \BuiltInTok{next}\NormalTok{(}\BuiltInTok{iter}\NormalTok{(train\_dl))}
\NormalTok{Ls, abs\_ }\OperatorTok{=}\NormalTok{ data[}\StringTok{\textquotesingle{}L\textquotesingle{}}\NormalTok{], data[}\StringTok{\textquotesingle{}ab\textquotesingle{}}\NormalTok{]}
\BuiltInTok{print}\NormalTok{(Ls.shape, abs\_.shape)}
\end{Highlighting}
\end{Shaded}

\begin{verbatim}
torch.Size([16, 1, 224, 224]) torch.Size([16, 2, 224, 224])
\end{verbatim}

\hypertarget{loss-functions}{%
\subsubsection{Loss functions}\label{loss-functions}}

The loss function of the discriminator for each image is the binary
cross entropy between the predictions and the ground truth: \emph{real}
if the real \textbf{ab} channels were fed into the discriminator of
\emph{fake} if the generated \textbf{ab} channels were used instead. The
loss function of the generator was the combination of the L1 error and
the loss function of the discriminator as it were \emph{real}
\textbf{ab} channels. The intuition behind is that the generator
``wins'' every time it fools the discriminator into assigning
\emph{real} predictions to its outputs.

\begin{Shaded}
\begin{Highlighting}[]
\KeywordTok{class}\NormalTok{ GANLoss(nn.Module):}
    \KeywordTok{def} \FunctionTok{\_\_init\_\_}\NormalTok{(}\VariableTok{self}\NormalTok{, gan\_mode}\OperatorTok{=}\StringTok{\textquotesingle{}vanilla\textquotesingle{}}\NormalTok{, real\_label}\OperatorTok{=}\FloatTok{1.0}\NormalTok{, fake\_label}\OperatorTok{=}\FloatTok{0.0}\NormalTok{):}
        \BuiltInTok{super}\NormalTok{().}\FunctionTok{\_\_init\_\_}\NormalTok{()}
        \VariableTok{self}\NormalTok{.register\_buffer(}\StringTok{\textquotesingle{}real\_label\textquotesingle{}}\NormalTok{, torch.tensor(real\_label))}
        \VariableTok{self}\NormalTok{.register\_buffer(}\StringTok{\textquotesingle{}fake\_label\textquotesingle{}}\NormalTok{, torch.tensor(fake\_label))}
        \ControlFlowTok{if}\NormalTok{ gan\_mode }\OperatorTok{==} \StringTok{\textquotesingle{}vanilla\textquotesingle{}}\NormalTok{:}
            \VariableTok{self}\NormalTok{.loss }\OperatorTok{=}\NormalTok{ nn.BCEWithLogitsLoss()}
        \ControlFlowTok{elif}\NormalTok{ gan\_mode }\OperatorTok{==} \StringTok{\textquotesingle{}lsgan\textquotesingle{}}\NormalTok{:}
            \VariableTok{self}\NormalTok{.loss }\OperatorTok{=}\NormalTok{ nn.MSELoss()}
    
    \KeywordTok{def}\NormalTok{ get\_labels(}\VariableTok{self}\NormalTok{, preds, target\_is\_real):}
        \ControlFlowTok{if}\NormalTok{ target\_is\_real:}
\NormalTok{            labels }\OperatorTok{=} \VariableTok{self}\NormalTok{.real\_label}
        \ControlFlowTok{else}\NormalTok{:}
\NormalTok{            labels }\OperatorTok{=} \VariableTok{self}\NormalTok{.fake\_label}
        \ControlFlowTok{return}\NormalTok{ labels.expand\_as(preds)}
    
    \KeywordTok{def} \FunctionTok{\_\_call\_\_}\NormalTok{(}\VariableTok{self}\NormalTok{, preds, target\_is\_real):}
\NormalTok{        labels }\OperatorTok{=} \VariableTok{self}\NormalTok{.get\_labels(preds, target\_is\_real)}
\NormalTok{        loss }\OperatorTok{=} \VariableTok{self}\NormalTok{.loss(preds, labels) }
        \ControlFlowTok{return}\NormalTok{ loss}
\end{Highlighting}
\end{Shaded}

\begin{Shaded}
\begin{Highlighting}[]
\KeywordTok{def}\NormalTok{ init\_weights(net, init}\OperatorTok{=}\StringTok{\textquotesingle{}norm\textquotesingle{}}\NormalTok{, gain}\OperatorTok{=}\FloatTok{0.02}\NormalTok{):}
    
    \KeywordTok{def}\NormalTok{ init\_func(m):}
\NormalTok{        classname }\OperatorTok{=}\NormalTok{ m.\_\_class\_\_.}\VariableTok{\_\_name\_\_}
        \ControlFlowTok{if} \BuiltInTok{hasattr}\NormalTok{(m, }\StringTok{\textquotesingle{}weight\textquotesingle{}}\NormalTok{) }\KeywordTok{and} \StringTok{\textquotesingle{}Conv\textquotesingle{}} \KeywordTok{in}\NormalTok{ classname:}
            \ControlFlowTok{if}\NormalTok{ init }\OperatorTok{==} \StringTok{\textquotesingle{}norm\textquotesingle{}}\NormalTok{:}
\NormalTok{                nn.init.normal\_(m.weight.data, mean}\OperatorTok{=}\FloatTok{0.0}\NormalTok{, std}\OperatorTok{=}\NormalTok{gain)}
            \ControlFlowTok{elif}\NormalTok{ init }\OperatorTok{==} \StringTok{\textquotesingle{}xavier\textquotesingle{}}\NormalTok{:}
\NormalTok{                nn.init.xavier\_normal\_(m.weight.data, gain}\OperatorTok{=}\NormalTok{gain)}
            \ControlFlowTok{elif}\NormalTok{ init }\OperatorTok{==} \StringTok{\textquotesingle{}kaiming\textquotesingle{}}\NormalTok{:}
\NormalTok{                nn.init.kaiming\_normal\_(m.weight.data, a}\OperatorTok{=}\DecValTok{0}\NormalTok{, mode}\OperatorTok{=}\StringTok{\textquotesingle{}fan\_in\textquotesingle{}}\NormalTok{)}
            
            \ControlFlowTok{if} \BuiltInTok{hasattr}\NormalTok{(m, }\StringTok{\textquotesingle{}bias\textquotesingle{}}\NormalTok{) }\KeywordTok{and}\NormalTok{ m.bias }\KeywordTok{is} \KeywordTok{not} \VariableTok{None}\NormalTok{:}
\NormalTok{                nn.init.constant\_(m.bias.data, }\FloatTok{0.0}\NormalTok{)}
        \ControlFlowTok{elif} \StringTok{\textquotesingle{}BatchNorm2d\textquotesingle{}} \KeywordTok{in}\NormalTok{ classname:}
\NormalTok{            nn.init.normal\_(m.weight.data, }\FloatTok{1.}\NormalTok{, gain)}
\NormalTok{            nn.init.constant\_(m.bias.data, }\FloatTok{0.}\NormalTok{)}
            
\NormalTok{    net.}\BuiltInTok{apply}\NormalTok{(init\_func)}
    \BuiltInTok{print}\NormalTok{(}\SpecialStringTok{f"model initialized with }\SpecialCharTok{\{}\NormalTok{init}\SpecialCharTok{\}}\SpecialStringTok{ initialization"}\NormalTok{)}
    \ControlFlowTok{return}\NormalTok{ net}

\KeywordTok{def}\NormalTok{ init\_model(model, device):}
\NormalTok{    model }\OperatorTok{=}\NormalTok{ model.to(device)}
\NormalTok{    model }\OperatorTok{=}\NormalTok{ init\_weights(model)}
    \ControlFlowTok{return}\NormalTok{ model}
\end{Highlighting}
\end{Shaded}

\begin{Shaded}
\begin{Highlighting}[]
\KeywordTok{class}\NormalTok{ AverageMeter:}
    \KeywordTok{def} \FunctionTok{\_\_init\_\_}\NormalTok{(}\VariableTok{self}\NormalTok{):}
        \VariableTok{self}\NormalTok{.reset()}
        
    \KeywordTok{def}\NormalTok{ reset(}\VariableTok{self}\NormalTok{):}
        \VariableTok{self}\NormalTok{.count, }\VariableTok{self}\NormalTok{.avg, }\VariableTok{self}\NormalTok{.}\BuiltInTok{sum} \OperatorTok{=}\NormalTok{ [}\FloatTok{0.}\NormalTok{] }\OperatorTok{*} \DecValTok{3}
    
    \KeywordTok{def}\NormalTok{ update(}\VariableTok{self}\NormalTok{, val, count}\OperatorTok{=}\DecValTok{1}\NormalTok{):}
        \VariableTok{self}\NormalTok{.count }\OperatorTok{+=}\NormalTok{ count}
        \VariableTok{self}\NormalTok{.}\BuiltInTok{sum} \OperatorTok{+=}\NormalTok{ count }\OperatorTok{*}\NormalTok{ val}
        \VariableTok{self}\NormalTok{.avg }\OperatorTok{=} \VariableTok{self}\NormalTok{.}\BuiltInTok{sum} \OperatorTok{/} \VariableTok{self}\NormalTok{.count}

\KeywordTok{def}\NormalTok{ create\_loss\_meters():}
\NormalTok{    loss\_D\_fake }\OperatorTok{=}\NormalTok{ AverageMeter()}
\NormalTok{    loss\_D\_real }\OperatorTok{=}\NormalTok{ AverageMeter()}
\NormalTok{    loss\_D }\OperatorTok{=}\NormalTok{ AverageMeter()}
\NormalTok{    loss\_G\_GAN }\OperatorTok{=}\NormalTok{ AverageMeter()}
\NormalTok{    loss\_G\_L1 }\OperatorTok{=}\NormalTok{ AverageMeter()}
\NormalTok{    loss\_G }\OperatorTok{=}\NormalTok{ AverageMeter()}
    
    \ControlFlowTok{return}\NormalTok{ \{}\StringTok{\textquotesingle{}loss\_D\_fake\textquotesingle{}}\NormalTok{: loss\_D\_fake,}
            \StringTok{\textquotesingle{}loss\_D\_real\textquotesingle{}}\NormalTok{: loss\_D\_real,}
            \StringTok{\textquotesingle{}loss\_D\textquotesingle{}}\NormalTok{: loss\_D,}
            \StringTok{\textquotesingle{}loss\_G\_GAN\textquotesingle{}}\NormalTok{: loss\_G\_GAN,}
            \StringTok{\textquotesingle{}loss\_G\_L1\textquotesingle{}}\NormalTok{: loss\_G\_L1,}
            \StringTok{\textquotesingle{}loss\_G\textquotesingle{}}\NormalTok{: loss\_G\}}

\KeywordTok{def}\NormalTok{ update\_losses(model, loss\_meter\_dict, count):}
    \ControlFlowTok{for}\NormalTok{ loss\_name, loss\_meter }\KeywordTok{in}\NormalTok{ loss\_meter\_dict.items():}
\NormalTok{        loss }\OperatorTok{=} \BuiltInTok{getattr}\NormalTok{(model, loss\_name)}
\NormalTok{        loss\_meter.update(loss.item(), count}\OperatorTok{=}\NormalTok{count)}

\KeywordTok{def}\NormalTok{ lab\_to\_rgb(L, ab):}
    \CommentTok{"""}
\CommentTok{    Takes a batch of images}
\CommentTok{    """}
    
\NormalTok{    L }\OperatorTok{=}\NormalTok{ (L }\OperatorTok{+} \FloatTok{1.}\NormalTok{) }\OperatorTok{*} \FloatTok{50.}
\NormalTok{    ab }\OperatorTok{=}\NormalTok{ ab }\OperatorTok{*} \FloatTok{110.}
\NormalTok{    Lab }\OperatorTok{=}\NormalTok{ torch.cat([L, ab], dim}\OperatorTok{=}\DecValTok{1}\NormalTok{).permute(}\DecValTok{0}\NormalTok{, }\DecValTok{2}\NormalTok{, }\DecValTok{3}\NormalTok{, }\DecValTok{1}\NormalTok{).cpu().numpy()}
\NormalTok{    rgb\_imgs }\OperatorTok{=}\NormalTok{ []}
    \ControlFlowTok{for}\NormalTok{ img }\KeywordTok{in}\NormalTok{ Lab:}
\NormalTok{        img\_rgb }\OperatorTok{=}\NormalTok{ lab2rgb(img)}
\NormalTok{        rgb\_imgs.append(img\_rgb)}
    \ControlFlowTok{return}\NormalTok{ np.stack(rgb\_imgs, axis}\OperatorTok{=}\DecValTok{0}\NormalTok{)}
    
\KeywordTok{def}\NormalTok{ model\_eval(model, data):}
\NormalTok{    model.net\_G.}\BuiltInTok{eval}\NormalTok{()}
    \ControlFlowTok{with}\NormalTok{ torch.no\_grad():}
\NormalTok{        model.setup\_input(data)}
\NormalTok{        model.forward()}
\NormalTok{    model.net\_G.train()}
\NormalTok{    fake\_color }\OperatorTok{=}\NormalTok{ model.fake\_color.detach()}
\NormalTok{    real\_color }\OperatorTok{=}\NormalTok{ model.ab}
\NormalTok{    L }\OperatorTok{=}\NormalTok{ model.L}
\NormalTok{    fake\_imgs }\OperatorTok{=}\NormalTok{ lab\_to\_rgb(L, fake\_color)}
\NormalTok{    real\_imgs }\OperatorTok{=}\NormalTok{ lab\_to\_rgb(L, real\_color)}
    \ControlFlowTok{return}\NormalTok{ fake\_imgs, real\_imgs}
    
\KeywordTok{def}\NormalTok{ visualize(model, data, save}\OperatorTok{=}\VariableTok{True}\NormalTok{):}
\NormalTok{    model.net\_G.}\BuiltInTok{eval}\NormalTok{()}
    \ControlFlowTok{with}\NormalTok{ torch.no\_grad():}
\NormalTok{        model.setup\_input(data)}
\NormalTok{        model.forward()}
\NormalTok{    model.net\_G.train()}
\NormalTok{    fake\_color }\OperatorTok{=}\NormalTok{ model.fake\_color.detach()}
\NormalTok{    real\_color }\OperatorTok{=}\NormalTok{ model.ab}
\NormalTok{    L }\OperatorTok{=}\NormalTok{ model.L}
\NormalTok{    fake\_imgs }\OperatorTok{=}\NormalTok{ lab\_to\_rgb(L, fake\_color)}
\NormalTok{    real\_imgs }\OperatorTok{=}\NormalTok{ lab\_to\_rgb(L, real\_color)}
\NormalTok{    fig }\OperatorTok{=}\NormalTok{ plt.figure(figsize}\OperatorTok{=}\NormalTok{(}\DecValTok{15}\NormalTok{, }\DecValTok{8}\NormalTok{))}
    \ControlFlowTok{for}\NormalTok{ i }\KeywordTok{in} \BuiltInTok{range}\NormalTok{(}\DecValTok{5}\NormalTok{):}
\NormalTok{        ax }\OperatorTok{=}\NormalTok{ plt.subplot(}\DecValTok{3}\NormalTok{, }\DecValTok{5}\NormalTok{, i }\OperatorTok{+} \DecValTok{1}\NormalTok{)}
\NormalTok{        ax.imshow(L[i][}\DecValTok{0}\NormalTok{].cpu(), cmap}\OperatorTok{=}\StringTok{\textquotesingle{}gray\textquotesingle{}}\NormalTok{)}
\NormalTok{        ax.axis(}\StringTok{"off"}\NormalTok{)}
\NormalTok{        ax }\OperatorTok{=}\NormalTok{ plt.subplot(}\DecValTok{3}\NormalTok{, }\DecValTok{5}\NormalTok{, i }\OperatorTok{+} \DecValTok{1} \OperatorTok{+} \DecValTok{5}\NormalTok{)}
\NormalTok{        ax.imshow(fake\_imgs[i])}
\NormalTok{        ax.axis(}\StringTok{"off"}\NormalTok{)}
\NormalTok{        ax }\OperatorTok{=}\NormalTok{ plt.subplot(}\DecValTok{3}\NormalTok{, }\DecValTok{5}\NormalTok{, i }\OperatorTok{+} \DecValTok{1} \OperatorTok{+} \DecValTok{10}\NormalTok{)}
\NormalTok{        ax.imshow(real\_imgs[i])}
\NormalTok{        ax.axis(}\StringTok{"off"}\NormalTok{)}
\NormalTok{    plt.show()}
    \ControlFlowTok{if}\NormalTok{ save:}
\NormalTok{        fig.savefig(}\SpecialStringTok{f"output/colorization\_}\SpecialCharTok{\{}\NormalTok{time}\SpecialCharTok{.}\NormalTok{time()}\SpecialCharTok{\}}\SpecialStringTok{.png"}\NormalTok{)}
    \ControlFlowTok{return}\NormalTok{ fake\_imgs, real\_imgs}
        
\KeywordTok{def}\NormalTok{ log\_results(loss\_meter\_dict):}
    \ControlFlowTok{for}\NormalTok{ loss\_name, loss\_meter }\KeywordTok{in}\NormalTok{ loss\_meter\_dict.items():}
        \BuiltInTok{print}\NormalTok{(}\SpecialStringTok{f"}\SpecialCharTok{\{}\NormalTok{loss\_name}\SpecialCharTok{\}}\SpecialStringTok{: }\SpecialCharTok{\{}\NormalTok{loss\_meter}\SpecialCharTok{.}\NormalTok{avg}\SpecialCharTok{:.5f\}}\SpecialStringTok{"}\NormalTok{)}
\end{Highlighting}
\end{Shaded}

\hypertarget{model}{%
\subsubsection{Model}\label{model}}

\begin{Shaded}
\begin{Highlighting}[]
\KeywordTok{class}\NormalTok{ MainModel(nn.Module):}
    \KeywordTok{def} \FunctionTok{\_\_init\_\_}\NormalTok{(}\VariableTok{self}\NormalTok{, net\_G}\OperatorTok{=}\VariableTok{None}\NormalTok{, net\_D}\OperatorTok{=}\VariableTok{None}\NormalTok{, use\_ViT\_gen }\OperatorTok{=} \VariableTok{False}\NormalTok{, lr\_G}\OperatorTok{=}\FloatTok{2e{-}4}\NormalTok{, lr\_D}\OperatorTok{=}\FloatTok{2e{-}4}\NormalTok{, }
\NormalTok{                 beta1}\OperatorTok{=}\FloatTok{0.5}\NormalTok{, beta2}\OperatorTok{=}\FloatTok{0.999}\NormalTok{, lambda\_L1}\OperatorTok{=}\FloatTok{100.}\NormalTok{):}
        \BuiltInTok{super}\NormalTok{().}\FunctionTok{\_\_init\_\_}\NormalTok{()}
        
        \VariableTok{self}\NormalTok{.device }\OperatorTok{=}\NormalTok{ torch.device(}\StringTok{"cuda"} \ControlFlowTok{if}\NormalTok{ torch.cuda.is\_available() }\ControlFlowTok{else} \StringTok{"cpu"}\NormalTok{)}
        \VariableTok{self}\NormalTok{.lambda\_L1 }\OperatorTok{=}\NormalTok{ lambda\_L1}
        \VariableTok{self}\NormalTok{.use\_ViT\_gen }\OperatorTok{=}\NormalTok{ use\_ViT\_gen}
        
        \ControlFlowTok{if}\NormalTok{ net\_G }\KeywordTok{is} \VariableTok{None}\NormalTok{:}
            \ControlFlowTok{raise} \PreprocessorTok{NotImplementedError}
        \ControlFlowTok{else}\NormalTok{:}
            \VariableTok{self}\NormalTok{.net\_G }\OperatorTok{=}\NormalTok{ net\_G.to(}\VariableTok{self}\NormalTok{.device)}
        
        \ControlFlowTok{if}\NormalTok{ net\_D }\KeywordTok{is} \VariableTok{None}\NormalTok{:}
            \VariableTok{self}\NormalTok{.net\_D }\OperatorTok{=}\NormalTok{ init\_model(PatchDiscriminator(input\_c}\OperatorTok{=}\DecValTok{3}\NormalTok{, n\_down}\OperatorTok{=}\DecValTok{3}\NormalTok{, num\_filters}\OperatorTok{=}\DecValTok{64}\NormalTok{), }\VariableTok{self}\NormalTok{.device)}
        \ControlFlowTok{else}\NormalTok{:}
            \VariableTok{self}\NormalTok{.net\_D }\OperatorTok{=}\NormalTok{ net\_D.to(}\VariableTok{self}\NormalTok{.device)}
            
        \CommentTok{\#self.GANcriterion = GANLoss(gan\_mode=\textquotesingle{}vanilla\textquotesingle{}).to(self.device)  \# Original BCE Loss}
        \VariableTok{self}\NormalTok{.GANcriterion }\OperatorTok{=}\NormalTok{ GANLoss(gan\_mode}\OperatorTok{=}\StringTok{\textquotesingle{}lsgan\textquotesingle{}}\NormalTok{).to(}\VariableTok{self}\NormalTok{.device)  }\CommentTok{\# Final improvement with Least Square Error loss}
        \VariableTok{self}\NormalTok{.L1criterion }\OperatorTok{=}\NormalTok{ nn.L1Loss()}
        \VariableTok{self}\NormalTok{.opt\_G }\OperatorTok{=}\NormalTok{ optim.Adam(}\VariableTok{self}\NormalTok{.net\_G.parameters(), lr}\OperatorTok{=}\NormalTok{lr\_G, betas}\OperatorTok{=}\NormalTok{(beta1, beta2))}
        \VariableTok{self}\NormalTok{.opt\_D }\OperatorTok{=}\NormalTok{ optim.Adam(}\VariableTok{self}\NormalTok{.net\_D.parameters(), lr}\OperatorTok{=}\NormalTok{lr\_D, betas}\OperatorTok{=}\NormalTok{(beta1, beta2))}
    
    \KeywordTok{def}\NormalTok{ set\_requires\_grad(}\VariableTok{self}\NormalTok{, model, requires\_grad}\OperatorTok{=}\VariableTok{True}\NormalTok{):}
        \ControlFlowTok{for}\NormalTok{ p }\KeywordTok{in}\NormalTok{ model.parameters():}
\NormalTok{            p.requires\_grad }\OperatorTok{=}\NormalTok{ requires\_grad}
        
    \KeywordTok{def}\NormalTok{ setup\_input(}\VariableTok{self}\NormalTok{, data):}
        \VariableTok{self}\NormalTok{.L }\OperatorTok{=}\NormalTok{ data[}\StringTok{\textquotesingle{}L\textquotesingle{}}\NormalTok{].to(}\VariableTok{self}\NormalTok{.device)}
        \VariableTok{self}\NormalTok{.ab }\OperatorTok{=}\NormalTok{ data[}\StringTok{\textquotesingle{}ab\textquotesingle{}}\NormalTok{].to(}\VariableTok{self}\NormalTok{.device)}
        
    \KeywordTok{def}\NormalTok{ forward(}\VariableTok{self}\NormalTok{):}
        \ControlFlowTok{if}\NormalTok{ (}\VariableTok{self}\NormalTok{.use\_ViT\_gen }\OperatorTok{==} \VariableTok{True}\NormalTok{):}
\NormalTok{            outputs }\OperatorTok{=} \VariableTok{self}\NormalTok{.net\_G(}\VariableTok{self}\NormalTok{.L.repeat(}\DecValTok{1}\NormalTok{,}\DecValTok{3}\NormalTok{,}\DecValTok{1}\NormalTok{,}\DecValTok{1}\NormalTok{))}
            \VariableTok{self}\NormalTok{.fake\_color }\OperatorTok{=}\NormalTok{ outputs.logits}
        \ControlFlowTok{else}\NormalTok{:}
            \VariableTok{self}\NormalTok{.fake\_color }\OperatorTok{=} \VariableTok{self}\NormalTok{.net\_G(}\VariableTok{self}\NormalTok{.L)}
    
    \KeywordTok{def}\NormalTok{ backward\_D(}\VariableTok{self}\NormalTok{):}
\NormalTok{        fake\_image }\OperatorTok{=}\NormalTok{ torch.cat([}\VariableTok{self}\NormalTok{.L, }\VariableTok{self}\NormalTok{.fake\_color], dim}\OperatorTok{=}\DecValTok{1}\NormalTok{)}
\NormalTok{        fake\_preds }\OperatorTok{=} \VariableTok{self}\NormalTok{.net\_D(fake\_image.detach())}
        \VariableTok{self}\NormalTok{.loss\_D\_fake }\OperatorTok{=} \VariableTok{self}\NormalTok{.GANcriterion(fake\_preds, }\VariableTok{False}\NormalTok{)}
\NormalTok{        real\_image }\OperatorTok{=}\NormalTok{ torch.cat([}\VariableTok{self}\NormalTok{.L, }\VariableTok{self}\NormalTok{.ab], dim}\OperatorTok{=}\DecValTok{1}\NormalTok{)}
\NormalTok{        real\_preds }\OperatorTok{=} \VariableTok{self}\NormalTok{.net\_D(real\_image)}
        \VariableTok{self}\NormalTok{.loss\_D\_real }\OperatorTok{=} \VariableTok{self}\NormalTok{.GANcriterion(real\_preds, }\VariableTok{True}\NormalTok{)}
        \VariableTok{self}\NormalTok{.loss\_D }\OperatorTok{=}\NormalTok{ (}\VariableTok{self}\NormalTok{.loss\_D\_fake }\OperatorTok{+} \VariableTok{self}\NormalTok{.loss\_D\_real) }\OperatorTok{*} \FloatTok{0.5}
        \VariableTok{self}\NormalTok{.loss\_D.backward()}
        
    \KeywordTok{def}\NormalTok{ backward\_G(}\VariableTok{self}\NormalTok{):}
\NormalTok{        fake\_image }\OperatorTok{=}\NormalTok{ torch.cat([}\VariableTok{self}\NormalTok{.L, }\VariableTok{self}\NormalTok{.fake\_color], dim}\OperatorTok{=}\DecValTok{1}\NormalTok{)}
\NormalTok{        fake\_preds }\OperatorTok{=} \VariableTok{self}\NormalTok{.net\_D(fake\_image)}
        \VariableTok{self}\NormalTok{.loss\_G\_GAN }\OperatorTok{=} \VariableTok{self}\NormalTok{.GANcriterion(fake\_preds, }\VariableTok{True}\NormalTok{)}
        \VariableTok{self}\NormalTok{.loss\_G\_L1 }\OperatorTok{=} \VariableTok{self}\NormalTok{.L1criterion(}\VariableTok{self}\NormalTok{.fake\_color, }\VariableTok{self}\NormalTok{.ab) }\OperatorTok{*} \VariableTok{self}\NormalTok{.lambda\_L1}
        \VariableTok{self}\NormalTok{.loss\_G }\OperatorTok{=} \VariableTok{self}\NormalTok{.loss\_G\_GAN }\OperatorTok{+} \VariableTok{self}\NormalTok{.loss\_G\_L1}
        \VariableTok{self}\NormalTok{.loss\_G.backward()}
    
    \KeywordTok{def}\NormalTok{ optimize(}\VariableTok{self}\NormalTok{):}
        \VariableTok{self}\NormalTok{.forward()}
        \VariableTok{self}\NormalTok{.net\_D.train()}
        \VariableTok{self}\NormalTok{.set\_requires\_grad(}\VariableTok{self}\NormalTok{.net\_D, }\VariableTok{True}\NormalTok{)}
        \VariableTok{self}\NormalTok{.opt\_D.zero\_grad()}
        \VariableTok{self}\NormalTok{.backward\_D()}
        \VariableTok{self}\NormalTok{.opt\_D.step()}
        
        \VariableTok{self}\NormalTok{.net\_G.train()}
        \VariableTok{self}\NormalTok{.set\_requires\_grad(}\VariableTok{self}\NormalTok{.net\_D, }\VariableTok{False}\NormalTok{)}
        \VariableTok{self}\NormalTok{.opt\_G.zero\_grad()}
        \VariableTok{self}\NormalTok{.backward\_G()}
        \VariableTok{self}\NormalTok{.opt\_G.step()}
\end{Highlighting}
\end{Shaded}

\begin{Shaded}
\begin{Highlighting}[]
\KeywordTok{class}\NormalTok{ PatchDiscriminator(nn.Module):}
    \KeywordTok{def} \FunctionTok{\_\_init\_\_}\NormalTok{(}\VariableTok{self}\NormalTok{, input\_c, num\_filters}\OperatorTok{=}\DecValTok{64}\NormalTok{, n\_down}\OperatorTok{=}\DecValTok{3}\NormalTok{):}
        \BuiltInTok{super}\NormalTok{().}\FunctionTok{\_\_init\_\_}\NormalTok{()}
\NormalTok{        model }\OperatorTok{=}\NormalTok{ [}\VariableTok{self}\NormalTok{.get\_layers(input\_c, num\_filters, norm}\OperatorTok{=}\VariableTok{False}\NormalTok{)]}
\NormalTok{        model }\OperatorTok{+=}\NormalTok{ [}\VariableTok{self}\NormalTok{.get\_layers(num\_filters }\OperatorTok{*} \DecValTok{2} \OperatorTok{**}\NormalTok{ i, num\_filters }\OperatorTok{*} \DecValTok{2} \OperatorTok{**}\NormalTok{ (i }\OperatorTok{+} \DecValTok{1}\NormalTok{), s}\OperatorTok{=}\DecValTok{1} \ControlFlowTok{if}\NormalTok{ i }\OperatorTok{==}\NormalTok{ (n\_down}\OperatorTok{{-}}\DecValTok{1}\NormalTok{) }\ControlFlowTok{else} \DecValTok{2}\NormalTok{) }
                          \ControlFlowTok{for}\NormalTok{ i }\KeywordTok{in} \BuiltInTok{range}\NormalTok{(n\_down)] }\CommentTok{\# the \textquotesingle{}if\textquotesingle{} statement is taking care of not using}
                                                  \CommentTok{\# stride of 2 for the last block in this loop}
\NormalTok{        model }\OperatorTok{+=}\NormalTok{ [}\VariableTok{self}\NormalTok{.get\_layers(num\_filters }\OperatorTok{*} \DecValTok{2} \OperatorTok{**}\NormalTok{ n\_down, }\DecValTok{1}\NormalTok{, s}\OperatorTok{=}\DecValTok{1}\NormalTok{, norm}\OperatorTok{=}\VariableTok{False}\NormalTok{, act}\OperatorTok{=}\VariableTok{False}\NormalTok{)] }\CommentTok{\# Make sure to not use normalization or}
                                                                                             \CommentTok{\# activation for the last layer of the model}
        \VariableTok{self}\NormalTok{.model }\OperatorTok{=}\NormalTok{ nn.Sequential(}\OperatorTok{*}\NormalTok{model)                                                   }
        
    \KeywordTok{def}\NormalTok{ get\_layers(}\VariableTok{self}\NormalTok{, ni, nf, k}\OperatorTok{=}\DecValTok{4}\NormalTok{, s}\OperatorTok{=}\DecValTok{2}\NormalTok{, p}\OperatorTok{=}\DecValTok{1}\NormalTok{, norm}\OperatorTok{=}\VariableTok{True}\NormalTok{, act}\OperatorTok{=}\VariableTok{True}\NormalTok{): }\CommentTok{\# when needing to make some repeatitive blocks of layers,}
\NormalTok{        layers }\OperatorTok{=}\NormalTok{ [nn.Conv2d(ni, nf, k, s, p, bias}\OperatorTok{=}\KeywordTok{not}\NormalTok{ norm)]          }\CommentTok{\# it\textquotesingle{}s always helpful to make a separate method for that purpose}
        \ControlFlowTok{if}\NormalTok{ norm: layers }\OperatorTok{+=}\NormalTok{ [nn.BatchNorm2d(nf)]}
        \ControlFlowTok{if}\NormalTok{ act: layers }\OperatorTok{+=}\NormalTok{ [nn.LeakyReLU(}\FloatTok{0.2}\NormalTok{, }\VariableTok{True}\NormalTok{)]}
        \ControlFlowTok{return}\NormalTok{ nn.Sequential(}\OperatorTok{*}\NormalTok{layers)}
    
    \KeywordTok{def}\NormalTok{ forward(}\VariableTok{self}\NormalTok{, x):}
        \ControlFlowTok{return} \VariableTok{self}\NormalTok{.model(x)}
\end{Highlighting}
\end{Shaded}

\begin{Shaded}
\begin{Highlighting}[]
\KeywordTok{def}\NormalTok{ build\_res\_unet(n\_input}\OperatorTok{=}\DecValTok{1}\NormalTok{, n\_output}\OperatorTok{=}\DecValTok{2}\NormalTok{, size}\OperatorTok{=}\DecValTok{256}\NormalTok{):}
\NormalTok{    device }\OperatorTok{=}\NormalTok{ torch.device(}\StringTok{"cuda"} \ControlFlowTok{if}\NormalTok{ torch.cuda.is\_available() }\ControlFlowTok{else} \StringTok{"cpu"}\NormalTok{)}
\NormalTok{    body }\OperatorTok{=}\NormalTok{ create\_body(resnet18, pretrained}\OperatorTok{=}\VariableTok{True}\NormalTok{, n\_in}\OperatorTok{=}\NormalTok{n\_input, cut}\OperatorTok{={-}}\DecValTok{2}\NormalTok{)}
\NormalTok{    net\_G }\OperatorTok{=}\NormalTok{ DynamicUnet(body, n\_output, (size, size)).to(device)}
    \ControlFlowTok{return}\NormalTok{ net\_G}
\end{Highlighting}
\end{Shaded}

\begin{Shaded}
\begin{Highlighting}[]
\KeywordTok{def}\NormalTok{ build\_visiontransformer(n\_output}\OperatorTok{=}\DecValTok{900}\NormalTok{, size}\OperatorTok{=}\DecValTok{256}\NormalTok{):}
\NormalTok{    net\_d }\OperatorTok{=}\NormalTok{ visiontransformer(image\_size}\OperatorTok{=}\NormalTok{size)}
\NormalTok{    net\_d.heads }\OperatorTok{=}\NormalTok{ nn.Linear(}\DecValTok{768}\NormalTok{, n\_output)}
\NormalTok{    torch.nn.init.xavier\_uniform(net\_d.heads.weight)}
    \ControlFlowTok{return}\NormalTok{ net\_d}
\end{Highlighting}
\end{Shaded}

\begin{Shaded}
\begin{Highlighting}[]
\CommentTok{\# Build transformer based generator}
\CommentTok{\# https://huggingface.co/docs/transformers/model\_doc/vit}

\ImportTok{from}\NormalTok{ transformers }\ImportTok{import}\NormalTok{ ViTForMaskedImageModeling, ViTConfig}


\KeywordTok{def}\NormalTok{ build\_VTi\_generator():}
\NormalTok{    device }\OperatorTok{=}\NormalTok{ torch.device(}\StringTok{"cuda"} \ControlFlowTok{if}\NormalTok{ torch.cuda.is\_available() }\ControlFlowTok{else} \StringTok{"cpu"}\NormalTok{)}
\NormalTok{    model }\OperatorTok{=}\NormalTok{ ViTForMaskedImageModeling.from\_pretrained(}\StringTok{"google/vit{-}base{-}patch16{-}224{-}in21k"}\NormalTok{)}
\CommentTok{\#    config = ViTConfig(num\_channels=1)}
\CommentTok{\#    model = ViTForMaskedImageModeling(config)}
\CommentTok{\#    model.decoder = nn.Sequential(nn.Conv2d(768, 512, kernel\_size=(1, 1), stride=(1, 1)), nn.PixelShuffle(upscale\_factor=16))   \#\# version 1 channel}
\NormalTok{    model.decoder }\OperatorTok{=}\NormalTok{ nn.Sequential(nn.Conv2d(}\DecValTok{768}\NormalTok{, }\DecValTok{768}\NormalTok{, kernel\_size}\OperatorTok{=}\DecValTok{3}\NormalTok{, stride}\OperatorTok{=}\DecValTok{1}\NormalTok{, padding}\OperatorTok{=}\DecValTok{1}\NormalTok{), }\OperatorTok{\textbackslash{}}
\NormalTok{                                    nn.ReLU(inplace}\OperatorTok{=}\VariableTok{True}\NormalTok{),}
\NormalTok{                                    nn.Conv2d(}\DecValTok{768}\NormalTok{, }\DecValTok{768}\NormalTok{, kernel\_size}\OperatorTok{=}\DecValTok{3}\NormalTok{, stride}\OperatorTok{=}\DecValTok{1}\NormalTok{, padding}\OperatorTok{=}\DecValTok{1}\NormalTok{), }\OperatorTok{\textbackslash{}}
\NormalTok{                                    nn.ReLU(inplace}\OperatorTok{=}\VariableTok{True}\NormalTok{),}
\NormalTok{                                    nn.Conv2d(}\DecValTok{768}\NormalTok{, }\DecValTok{512}\NormalTok{, kernel\_size}\OperatorTok{=}\DecValTok{3}\NormalTok{, stride}\OperatorTok{=}\DecValTok{1}\NormalTok{, padding}\OperatorTok{=}\DecValTok{1}\NormalTok{), }\OperatorTok{\textbackslash{}}
\NormalTok{                                    nn.PixelShuffle(upscale\_factor}\OperatorTok{=}\DecValTok{16}\NormalTok{))}
\NormalTok{    model }\OperatorTok{=}\NormalTok{ model.to(device)}
    \ControlFlowTok{return}\NormalTok{ model}
\end{Highlighting}
\end{Shaded}

\begin{Shaded}
\begin{Highlighting}[]
\KeywordTok{def}\NormalTok{ pretrain\_generator(net\_G, train\_dl, opt, criterion, epochs):}
    \ControlFlowTok{for}\NormalTok{ e }\KeywordTok{in} \BuiltInTok{range}\NormalTok{(epochs):}
\NormalTok{        loss\_meter }\OperatorTok{=}\NormalTok{ AverageMeter()}
        \ControlFlowTok{for}\NormalTok{ data }\KeywordTok{in}\NormalTok{ tqdm(train\_dl):}
\NormalTok{            L, ab }\OperatorTok{=}\NormalTok{ data[}\StringTok{\textquotesingle{}L\textquotesingle{}}\NormalTok{].to(device), data[}\StringTok{\textquotesingle{}ab\textquotesingle{}}\NormalTok{].to(device)}
\NormalTok{            preds }\OperatorTok{=}\NormalTok{ net\_G(L.repeat(}\DecValTok{1}\NormalTok{,}\DecValTok{3}\NormalTok{,}\DecValTok{1}\NormalTok{,}\DecValTok{1}\NormalTok{))}
\NormalTok{            preds }\OperatorTok{=}\NormalTok{ preds.logits}
\NormalTok{            loss }\OperatorTok{=}\NormalTok{ criterion(preds, ab)}
\NormalTok{            opt.zero\_grad()}
\NormalTok{            loss.backward()}
\NormalTok{            opt.step()}
            
\NormalTok{            loss\_meter.update(loss.item(), L.size(}\DecValTok{0}\NormalTok{))}
            
        \BuiltInTok{print}\NormalTok{(}\SpecialStringTok{f"Epoch }\SpecialCharTok{\{}\NormalTok{e }\OperatorTok{+} \DecValTok{1}\SpecialCharTok{\}}\SpecialStringTok{/}\SpecialCharTok{\{}\NormalTok{epochs}\SpecialCharTok{\}}\SpecialStringTok{"}\NormalTok{)}
        \BuiltInTok{print}\NormalTok{(}\SpecialStringTok{f"L1 Loss: }\SpecialCharTok{\{}\NormalTok{loss\_meter}\SpecialCharTok{.}\NormalTok{avg}\SpecialCharTok{:.5f\}}\SpecialStringTok{"}\NormalTok{)}
        \CommentTok{\#torch.save(net\_G.state\_dict(), \textquotesingle{}models/net\_G\_resnet18\_model{-}\textquotesingle{} + str(e) +\textquotesingle{}.pt\textquotesingle{})}
\end{Highlighting}
\end{Shaded}

\begin{Shaded}
\begin{Highlighting}[]
\KeywordTok{def}\NormalTok{ train\_model(model, train\_dl, epochs, display\_every}\OperatorTok{=}\DecValTok{200}\NormalTok{, first\_epoch}\OperatorTok{=}\DecValTok{0}\NormalTok{):}
\NormalTok{    data\_val }\OperatorTok{=} \BuiltInTok{next}\NormalTok{(}\BuiltInTok{iter}\NormalTok{(val\_dl)) }\CommentTok{\# getting a batch for visualizing the model output after fixed intervals}
    \ControlFlowTok{for}\NormalTok{ e }\KeywordTok{in} \BuiltInTok{range}\NormalTok{(first\_epoch, epochs):}
\NormalTok{        loss\_meter\_dict }\OperatorTok{=}\NormalTok{ create\_loss\_meters() }\CommentTok{\# function returing a dictionary of objects to }
\NormalTok{        i }\OperatorTok{=} \DecValTok{0}                                  \CommentTok{\# log the losses of the complete network}
        \ControlFlowTok{for}\NormalTok{ data }\KeywordTok{in}\NormalTok{ tqdm(train\_dl):}
\NormalTok{            model.setup\_input(data) }
\NormalTok{            model.optimize()}
\NormalTok{            update\_losses(model, loss\_meter\_dict, count}\OperatorTok{=}\NormalTok{data[}\StringTok{\textquotesingle{}L\textquotesingle{}}\NormalTok{].size(}\DecValTok{0}\NormalTok{)) }\CommentTok{\# function updating the log objects}
\NormalTok{            i }\OperatorTok{+=} \DecValTok{1}
            \ControlFlowTok{if}\NormalTok{ i }\OperatorTok{\%}\NormalTok{ display\_every }\OperatorTok{==} \DecValTok{0}\NormalTok{:}
                \BuiltInTok{print}\NormalTok{(}\SpecialStringTok{f"}\CharTok{\textbackslash{}n}\SpecialStringTok{Epoch }\SpecialCharTok{\{}\NormalTok{e}\OperatorTok{+}\DecValTok{1}\SpecialCharTok{\}}\SpecialStringTok{/}\SpecialCharTok{\{}\NormalTok{epochs}\SpecialCharTok{\}}\SpecialStringTok{"}\NormalTok{)}
                \BuiltInTok{print}\NormalTok{(}\SpecialStringTok{f"Iteration }\SpecialCharTok{\{}\NormalTok{i}\SpecialCharTok{\}}\SpecialStringTok{/}\SpecialCharTok{\{}\BuiltInTok{len}\NormalTok{(train\_dl)}\SpecialCharTok{\}}\SpecialStringTok{"}\NormalTok{)}
\NormalTok{                log\_results(loss\_meter\_dict) }\CommentTok{\# function to print out the losses}
\NormalTok{                visualize(model, data\_val, save}\OperatorTok{=}\VariableTok{False}\NormalTok{) }\CommentTok{\# function displaying the model\textquotesingle{}s outputs}
\NormalTok{        torch.save(model.state\_dict(), }\StringTok{\textquotesingle{}models/model6{-}3channel{-}ViT.pt\textquotesingle{}}\NormalTok{)}

\end{Highlighting}
\end{Shaded}

\hypertarget{metrics}{%
\subsubsection{Metrics}\label{metrics}}

To assess the different networks architectures we selected a set of
metrics for regression models:

\begin{itemize}
\tightlist
\item
  Correlation coeficient \emph{R} squared
\item
  Explained variance
\item
  Mean absolute error
\item
  Median absolute error
\item
  Mean squared error
\end{itemize}

We calculated all these metrics for each one of the \textbf{ab}
channels.

\begin{Shaded}
\begin{Highlighting}[]
\ImportTok{from}\NormalTok{ sklearn.metrics }\ImportTok{import}\NormalTok{ r2\_score, explained\_variance\_score, mean\_absolute\_error, mean\_squared\_error, median\_absolute\_error}
\ImportTok{import}\NormalTok{ pandas }\ImportTok{as}\NormalTok{ pd}

\KeywordTok{def}\NormalTok{ get\_metrics(model, dl):}
\NormalTok{    test\_iter }\OperatorTok{=} \BuiltInTok{iter}\NormalTok{(dl)}
\NormalTok{    [fa,fb,ra,rb] }\OperatorTok{=}\NormalTok{ [np.zeros((}\DecValTok{1}\NormalTok{,SIZE}\OperatorTok{**}\DecValTok{2}\NormalTok{))] }\OperatorTok{*} \DecValTok{4}

    \ControlFlowTok{for}\NormalTok{ data }\KeywordTok{in}\NormalTok{ tqdm(test\_iter):}
\NormalTok{        fake\_imgs, real\_imgs }\OperatorTok{=}\NormalTok{ model\_eval(model, data)}
\NormalTok{        fab }\OperatorTok{=}\NormalTok{ np.reshape(fake\_imgs[:,...,}\DecValTok{1}\NormalTok{],(fake\_imgs.shape[}\DecValTok{0}\NormalTok{],}\OperatorTok{{-}}\DecValTok{1}\NormalTok{))}
\NormalTok{        rab }\OperatorTok{=}\NormalTok{ np.reshape(real\_imgs[:,...,}\DecValTok{1}\NormalTok{],(real\_imgs.shape[}\DecValTok{0}\NormalTok{],}\OperatorTok{{-}}\DecValTok{1}\NormalTok{))}
\NormalTok{        fbb }\OperatorTok{=}\NormalTok{ np.reshape(fake\_imgs[:,...,}\DecValTok{2}\NormalTok{],(fake\_imgs.shape[}\DecValTok{0}\NormalTok{],}\OperatorTok{{-}}\DecValTok{1}\NormalTok{))}
\NormalTok{        rbb }\OperatorTok{=}\NormalTok{ np.reshape(real\_imgs[:,...,}\DecValTok{2}\NormalTok{],(real\_imgs.shape[}\DecValTok{0}\NormalTok{],}\OperatorTok{{-}}\DecValTok{1}\NormalTok{))}
\NormalTok{        fa }\OperatorTok{=}\NormalTok{ np.concatenate((fa,fab))}
\NormalTok{        fb }\OperatorTok{=}\NormalTok{ np.concatenate((fb,fbb))}
\NormalTok{        ra }\OperatorTok{=}\NormalTok{ np.concatenate((ra,rab))}
\NormalTok{        rb }\OperatorTok{=}\NormalTok{ np.concatenate((rb,rbb))}

\NormalTok{    table}\OperatorTok{=}\NormalTok{[]}
\NormalTok{    table.append([}\StringTok{"R{-}square"}\NormalTok{, r2\_score(fa, ra), r2\_score(fb, rb)])}
\NormalTok{    table.append([}\StringTok{"Explained variance"}\NormalTok{, explained\_variance\_score(fa, ra), explained\_variance\_score(fb, rb)])}
\NormalTok{    table.append([}\StringTok{"Mean absolute error"}\NormalTok{, mean\_absolute\_error(fa, ra), mean\_absolute\_error(fb, rb)])}
\NormalTok{    table.append([}\StringTok{"Median absolute error"}\NormalTok{, median\_absolute\_error(fa, ra), median\_absolute\_error(fb, rb)])}
\NormalTok{    table.append([}\StringTok{"Mean squared error"}\NormalTok{, mean\_squared\_error(fa, ra), mean\_squared\_error(fb, rb)])}
\NormalTok{    table.append([}\StringTok{"Sample size"}\NormalTok{, fa.shape[}\DecValTok{0}\NormalTok{], fb.shape[}\DecValTok{0}\NormalTok{]])}

\NormalTok{    df }\OperatorTok{=}\NormalTok{ pd.DataFrame(np.array(table),columns }\OperatorTok{=}\NormalTok{ [}\StringTok{\textquotesingle{}Metric\textquotesingle{}}\NormalTok{,}\StringTok{\textquotesingle{}a{-}channel\textquotesingle{}}\NormalTok{,}\StringTok{\textquotesingle{}b{-}channel\textquotesingle{}}\NormalTok{])}
    \ControlFlowTok{return}\NormalTok{ df}
\end{Highlighting}
\end{Shaded}

\hypertarget{first-approach-resnet18-generator}{%
\subsubsection{\texorpdfstring{First approach: \emph{ResNet18}
generator}{First approach: ResNet18 generator}}\label{first-approach-resnet18-generator}}

One of the challenges of GANs is that, at the beginning of the training,
the task of the discriminator is much easier than that of the generator
because the generated outputs are very different from the real ones. In
this situation, the discriminator learns so much faster and gives no
time to the generator to adapt. To avoid this, we gave the generator a
\emph{head start} by training it alone (without the generator) for 20
epochs with a L1 loss function and saving its weights.

\begin{Shaded}
\begin{Highlighting}[]
\CommentTok{\# Pretraining of ResNet18 Generator}

\NormalTok{net\_G }\OperatorTok{=}\NormalTok{ build\_res\_unet(n\_input}\OperatorTok{=}\DecValTok{1}\NormalTok{, n\_output}\OperatorTok{=}\DecValTok{2}\NormalTok{, size}\OperatorTok{=}\DecValTok{256}\NormalTok{)}
\NormalTok{opt }\OperatorTok{=}\NormalTok{ optim.Adam(net\_G.parameters(), lr}\OperatorTok{=}\FloatTok{1e{-}4}\NormalTok{)}
\NormalTok{criterion }\OperatorTok{=}\NormalTok{ nn.L1Loss()        }
\NormalTok{pretrain\_generator(net\_G, train\_dl, opt, criterion, }\DecValTok{20}\NormalTok{)}
\NormalTok{torch.save(net\_G.state\_dict(), }\StringTok{"models/net\_G\_resnet18\_model{-}19.pt"}\NormalTok{)}
\end{Highlighting}
\end{Shaded}

After that we started the parallel training of the generator and the
patch discriminator for another 20 epochs.

\begin{Shaded}
\begin{Highlighting}[]
\NormalTok{net\_G }\OperatorTok{=}\NormalTok{ build\_res\_unet(n\_input}\OperatorTok{=}\DecValTok{1}\NormalTok{, n\_output}\OperatorTok{=}\DecValTok{2}\NormalTok{, size}\OperatorTok{=}\DecValTok{256}\NormalTok{)}
\NormalTok{net\_G.load\_state\_dict(torch.load(}\StringTok{"models/net\_G\_resnet18\_model{-}19.pt"}\NormalTok{, map\_location}\OperatorTok{=}\NormalTok{device))}
\NormalTok{model }\OperatorTok{=}\NormalTok{ MainModel(net\_G}\OperatorTok{=}\NormalTok{net\_G)}
\NormalTok{train\_model(model, train\_dl, }\DecValTok{20}\NormalTok{)}
\NormalTok{torch.save(model.state\_dict(), }\StringTok{"models/colorization2{-}epoch20.pt"}\NormalTok{)}
\end{Highlighting}
\end{Shaded}

\begin{Shaded}
\begin{Highlighting}[]
\NormalTok{net\_G }\OperatorTok{=}\NormalTok{ build\_res\_unet(n\_input}\OperatorTok{=}\DecValTok{1}\NormalTok{, n\_output}\OperatorTok{=}\DecValTok{2}\NormalTok{, size}\OperatorTok{=}\DecValTok{224}\NormalTok{)}
\NormalTok{model }\OperatorTok{=}\NormalTok{ MainModel(net\_G }\OperatorTok{=}\NormalTok{ net\_G)}
\NormalTok{model.load\_state\_dict(torch.load(}\StringTok{"models/colorization2{-}epoch20.pt"}\NormalTok{, map\_location}\OperatorTok{=}\NormalTok{device))}
\end{Highlighting}
\end{Shaded}

\begin{verbatim}
model initialized with norm initialization





<All keys matched successfully>
\end{verbatim}

\begin{Shaded}
\begin{Highlighting}[]
\NormalTok{my\_paths }\OperatorTok{=}\NormalTok{ glob.glob(}\StringTok{"images/*.JPG"}\NormalTok{)}
\NormalTok{my\_dl }\OperatorTok{=}\NormalTok{ make\_dataloaders(paths}\OperatorTok{=}\NormalTok{my\_paths, split}\OperatorTok{=}\StringTok{\textquotesingle{}val\textquotesingle{}}\NormalTok{,  n\_workers}\OperatorTok{=}\DecValTok{0}\NormalTok{) }
\NormalTok{my\_iter }\OperatorTok{=} \BuiltInTok{iter}\NormalTok{(my\_dl)}
\end{Highlighting}
\end{Shaded}

\begin{Shaded}
\begin{Highlighting}[]
\NormalTok{data }\OperatorTok{=} \BuiltInTok{next}\NormalTok{(my\_iter)}
\NormalTok{fake\_imgs, real\_imgs }\OperatorTok{=}\NormalTok{ visualize(model, data, }\VariableTok{False}\NormalTok{)}
\end{Highlighting}
\end{Shaded}

\begin{figure}
\centering
\includegraphics{Project_files/Project_33_0.png}
\caption{png}
\end{figure}

\begin{Shaded}
\begin{Highlighting}[]
\NormalTok{df }\OperatorTok{=}\NormalTok{ get\_metrics(model, my\_dl)}
\end{Highlighting}
\end{Shaded}

\begin{Shaded}
\begin{Highlighting}[]
\NormalTok{df.to\_latex(caption}\OperatorTok{=}\NormalTok{(}\StringTok{"ResNet18 metrics on personal photos"}\NormalTok{,}\StringTok{"Metrics of ResNet18 generator on dataset of personal photos"}\NormalTok{))}
\end{Highlighting}
\end{Shaded}

\begin{Shaded}
\begin{Highlighting}[]
\NormalTok{df}
\end{Highlighting}
\end{Shaded}

\begin{Shaded}
\begin{Highlighting}[]
\NormalTok{df }\OperatorTok{=}\NormalTok{ get\_metrics(model, val\_dl)}
\end{Highlighting}
\end{Shaded}

\begin{Shaded}
\begin{Highlighting}[]
\NormalTok{df.to\_latex(caption}\OperatorTok{=}\NormalTok{(}\StringTok{"ResNet18 metrics on validation dataset"}\NormalTok{,}\StringTok{"Metrics of ResNet18 generator on validation dataset"}\NormalTok{))}
\end{Highlighting}
\end{Shaded}

\begin{table}\centering\caption[Metrics of ResNet18 generator on validation dataset]{ResNet18 metrics on validation dataset}\begin{tabular}{llll}\toprule{} &                 Metric &              a-channel &             b-channel \\\midrule0 &               R-square &     0.9762806708182279 &    0.7944829539176959 \\1 &     Explained variance &     0.9763600120897222 &    0.8047133044855702 \\2 &    Mean absolute error &   0.022080948550583485 &   0.08126157218579984 \\3 &  Median absolute error &   0.011479740269680015 &  0.053269025524179094 \\4 &     Mean squared error &  0.0016633580971493562 &  0.014459893084060526 \\5 &            Sample size &                   2000 &                  2000 \\\bottomrule\end{tabular}\end{table}

\begin{Shaded}
\begin{Highlighting}[]
\NormalTok{df.to\_pickle(}\StringTok{\textquotesingle{}results/ResNet18onValidation.pkl\textquotesingle{}}\NormalTok{)}
\end{Highlighting}
\end{Shaded}

The results of the model with the ResNet18 generator are acceptable.
However, many times they do not look natural because of an excessive use
of colors by the generator that resulted in colorful blotches in the
pictures.

In agreement with the visual inspection, the resulting metrics on the
validation dataset showed that the network did a pretty good job at
predicting the \textbf{a}-channel with over 97\% of the variance of the
channel predicted by the model with a very low mean squared error.
However, the prediction on the \textbf{b}-channel was not as good with
the model predicting only 80\% of its variance.

\hypertarget{second-approach-vit-as-discriminator}{%
\subsubsection{Second approach: ViT as
discriminator}\label{second-approach-vit-as-discriminator}}

\begin{Shaded}
\begin{Highlighting}[]
\CommentTok{\#net\_G = build\_res\_unet(n\_input=1, n\_output=2, size=256)}
\CommentTok{\#net\_G.load\_state\_dict(torch.load("models/net\_G\_resnet18\_model{-}19.pt", map\_location=device))}
\CommentTok{\#net\_D = build\_visiontransformer()}
\CommentTok{\#model = MainModel(net\_G=net\_G, net\_D=net\_D)}
\CommentTok{\#model.load\_state\_dict(torch.load(\textquotesingle{}models/model2{-}1.pt\textquotesingle{}))}
\CommentTok{\#train\_model(model, train\_dl, 20, first\_epoch=1)}
\CommentTok{\#torch.save(net\_G.state\_dict(), "models/colorization3{-}epoch20.pt")}
\end{Highlighting}
\end{Shaded}

\begin{Shaded}
\begin{Highlighting}[]
\NormalTok{net\_G }\OperatorTok{=}\NormalTok{ build\_VTi\_generator()}
\NormalTok{opt }\OperatorTok{=}\NormalTok{ optim.Adam(net\_G.parameters(), lr}\OperatorTok{=}\FloatTok{1e{-}4}\NormalTok{)}
\NormalTok{criterion }\OperatorTok{=}\NormalTok{ nn.L1Loss()        }
\CommentTok{\#pretrain\_generator(net\_G, train\_dl, opt, criterion, 20)}
\CommentTok{\#torch.save(net\_G.state\_dict(), "models/net\_G\_ViT{-}20{-}pretraining.pt")}
\NormalTok{net\_G.load\_state\_dict(torch.load(}\StringTok{"models/net\_G\_ViT{-}20{-}pretraining.pt"}\NormalTok{, map\_location}\OperatorTok{=}\NormalTok{device))}
\NormalTok{model }\OperatorTok{=}\NormalTok{ MainModel(net\_G }\OperatorTok{=}\NormalTok{ net\_G, use\_ViT\_gen}\OperatorTok{=}\VariableTok{True}\NormalTok{)}
\NormalTok{train\_model(model, train\_dl, }\DecValTok{20}\NormalTok{)}
\NormalTok{torch.save(net\_G.state\_dict(), }\StringTok{"models/colorization6{-}ViT{-}epoch20.pt"}\NormalTok{)}
\end{Highlighting}
\end{Shaded}

\begin{Shaded}
\begin{Highlighting}[]
\NormalTok{net\_G }\OperatorTok{=}\NormalTok{ build\_VTi\_generator()}
\CommentTok{\#model = MainModel(net\_G = net\_G, use\_ViT\_gen=True)}
\CommentTok{\#net\_G = build\_res\_unet(n\_input=1, n\_output=2, size=224)}
\NormalTok{model }\OperatorTok{=}\NormalTok{ MainModel(net\_G }\OperatorTok{=}\NormalTok{ net\_G, use\_ViT\_gen}\OperatorTok{=}\VariableTok{True}\NormalTok{)}
\CommentTok{\# net\_G.load\_state\_dict(torch.load("res18{-}unet.pt", map\_location=device))}
\CommentTok{\#model = MainModel(net\_G=net\_G, net\_D = build\_visiontransformer())}
\NormalTok{net\_G.load\_state\_dict(torch.load(}\StringTok{"models/colorization6{-}ViT{-}epoch20.pt"}\NormalTok{, map\_location}\OperatorTok{=}\NormalTok{device))}
\NormalTok{test\_paths }\OperatorTok{=}\NormalTok{ glob.glob(}\StringTok{"images/*.JPG"}\NormalTok{)}

\NormalTok{my\_dl }\OperatorTok{=}\NormalTok{ make\_dataloaders(paths}\OperatorTok{=}\NormalTok{test\_paths, split}\OperatorTok{=}\StringTok{\textquotesingle{}val\textquotesingle{}}\NormalTok{,  n\_workers}\OperatorTok{=}\DecValTok{0}\NormalTok{)  }\CommentTok{\#\# n\_workers=0 in Windows}
\end{Highlighting}
\end{Shaded}

\begin{verbatim}
Some weights of the model checkpoint at google/vit-base-patch16-224-in21k were not used when initializing ViTForMaskedImageModeling: ['pooler.dense.bias', 'pooler.dense.weight']
- This IS expected if you are initializing ViTForMaskedImageModeling from the checkpoint of a model trained on another task or with another architecture (e.g. initializing a BertForSequenceClassification model from a BertForPreTraining model).
- This IS NOT expected if you are initializing ViTForMaskedImageModeling from the checkpoint of a model that you expect to be exactly identical (initializing a BertForSequenceClassification model from a BertForSequenceClassification model).
Some weights of ViTForMaskedImageModeling were not initialized from the model checkpoint at google/vit-base-patch16-224-in21k and are newly initialized: ['embeddings.mask_token', 'decoder.0.weight', 'decoder.0.bias']
You should probably TRAIN this model on a down-stream task to be able to use it for predictions and inference.


model initialized with norm initialization


c:\users\fnoya\appdata\local\programs\python\python37\lib\site-packages\ipykernel_launcher.py:10: DeprecationWarning: BICUBIC is deprecated and will be removed in Pillow 10 (2023-07-01). Use Resampling.BICUBIC instead.
  # Remove the CWD from sys.path while we load stuff.
\end{verbatim}

\begin{Shaded}
\begin{Highlighting}[]
\NormalTok{df }\OperatorTok{=}\NormalTok{ get\_metrics(model, val\_dl)}
\NormalTok{df}
\end{Highlighting}
\end{Shaded}

\begin{verbatim}
  0%|          | 0/125 [00:00<?, ?it/s]


c:\users\fnoya\appdata\local\programs\python\python37\lib\site-packages\skimage\_shared\utils.py:394: UserWarning: Color data out of range: Z < 0 in 1 pixels
  return func(*args, **kwargs)
c:\users\fnoya\appdata\local\programs\python\python37\lib\site-packages\skimage\_shared\utils.py:394: UserWarning: Color data out of range: Z < 0 in 2 pixels
  return func(*args, **kwargs)
c:\users\fnoya\appdata\local\programs\python\python37\lib\site-packages\skimage\_shared\utils.py:394: UserWarning: Color data out of range: Z < 0 in 12 pixels
  return func(*args, **kwargs)
c:\users\fnoya\appdata\local\programs\python\python37\lib\site-packages\skimage\_shared\utils.py:394: UserWarning: Color data out of range: Z < 0 in 3 pixels
  return func(*args, **kwargs)
c:\users\fnoya\appdata\local\programs\python\python37\lib\site-packages\skimage\_shared\utils.py:394: UserWarning: Color data out of range: Z < 0 in 100 pixels
  return func(*args, **kwargs)
c:\users\fnoya\appdata\local\programs\python\python37\lib\site-packages\skimage\_shared\utils.py:394: UserWarning: Color data out of range: Z < 0 in 19 pixels
  return func(*args, **kwargs)
c:\users\fnoya\appdata\local\programs\python\python37\lib\site-packages\skimage\_shared\utils.py:394: UserWarning: Color data out of range: Z < 0 in 4 pixels
  return func(*args, **kwargs)
c:\users\fnoya\appdata\local\programs\python\python37\lib\site-packages\skimage\_shared\utils.py:394: UserWarning: Color data out of range: Z < 0 in 37 pixels
  return func(*args, **kwargs)
c:\users\fnoya\appdata\local\programs\python\python37\lib\site-packages\skimage\_shared\utils.py:394: UserWarning: Color data out of range: Z < 0 in 66 pixels
  return func(*args, **kwargs)
c:\users\fnoya\appdata\local\programs\python\python37\lib\site-packages\skimage\_shared\utils.py:394: UserWarning: Color data out of range: Z < 0 in 14 pixels
  return func(*args, **kwargs)
c:\users\fnoya\appdata\local\programs\python\python37\lib\site-packages\skimage\_shared\utils.py:394: UserWarning: Color data out of range: Z < 0 in 8 pixels
  return func(*args, **kwargs)
c:\users\fnoya\appdata\local\programs\python\python37\lib\site-packages\skimage\_shared\utils.py:394: UserWarning: Color data out of range: Z < 0 in 20 pixels
  return func(*args, **kwargs)
c:\users\fnoya\appdata\local\programs\python\python37\lib\site-packages\skimage\_shared\utils.py:394: UserWarning: Color data out of range: Z < 0 in 11 pixels
  return func(*args, **kwargs)
c:\users\fnoya\appdata\local\programs\python\python37\lib\site-packages\skimage\_shared\utils.py:394: UserWarning: Color data out of range: Z < 0 in 9 pixels
  return func(*args, **kwargs)
c:\users\fnoya\appdata\local\programs\python\python37\lib\site-packages\skimage\_shared\utils.py:394: UserWarning: Color data out of range: Z < 0 in 60 pixels
  return func(*args, **kwargs)
c:\users\fnoya\appdata\local\programs\python\python37\lib\site-packages\skimage\_shared\utils.py:394: UserWarning: Color data out of range: Z < 0 in 6 pixels
  return func(*args, **kwargs)
c:\users\fnoya\appdata\local\programs\python\python37\lib\site-packages\skimage\_shared\utils.py:394: UserWarning: Color data out of range: Z < 0 in 5 pixels
  return func(*args, **kwargs)
c:\users\fnoya\appdata\local\programs\python\python37\lib\site-packages\skimage\_shared\utils.py:394: UserWarning: Color data out of range: Z < 0 in 123 pixels
  return func(*args, **kwargs)
c:\users\fnoya\appdata\local\programs\python\python37\lib\site-packages\skimage\_shared\utils.py:394: UserWarning: Color data out of range: Z < 0 in 21 pixels
  return func(*args, **kwargs)
c:\users\fnoya\appdata\local\programs\python\python37\lib\site-packages\skimage\_shared\utils.py:394: UserWarning: Color data out of range: Z < 0 in 45 pixels
  return func(*args, **kwargs)
c:\users\fnoya\appdata\local\programs\python\python37\lib\site-packages\skimage\_shared\utils.py:394: UserWarning: Color data out of range: Z < 0 in 121 pixels
  return func(*args, **kwargs)
c:\users\fnoya\appdata\local\programs\python\python37\lib\site-packages\skimage\_shared\utils.py:394: UserWarning: Color data out of range: Z < 0 in 47 pixels
  return func(*args, **kwargs)
c:\users\fnoya\appdata\local\programs\python\python37\lib\site-packages\skimage\_shared\utils.py:394: UserWarning: Color data out of range: Z < 0 in 10 pixels
  return func(*args, **kwargs)
c:\users\fnoya\appdata\local\programs\python\python37\lib\site-packages\skimage\_shared\utils.py:394: UserWarning: Color data out of range: Z < 0 in 17 pixels
  return func(*args, **kwargs)
c:\users\fnoya\appdata\local\programs\python\python37\lib\site-packages\skimage\_shared\utils.py:394: UserWarning: Color data out of range: Z < 0 in 38 pixels
  return func(*args, **kwargs)
c:\users\fnoya\appdata\local\programs\python\python37\lib\site-packages\skimage\_shared\utils.py:394: UserWarning: Color data out of range: Z < 0 in 16 pixels
  return func(*args, **kwargs)
c:\users\fnoya\appdata\local\programs\python\python37\lib\site-packages\skimage\_shared\utils.py:394: UserWarning: Color data out of range: Z < 0 in 13 pixels
  return func(*args, **kwargs)
\end{verbatim}

Metric

a-channel

b-channel

0

R-square

0.9762806708182279

0.7944829539176959

1

Explained variance

0.9763600120897222

0.8047133044855702

2

Mean absolute error

0.022080948550583485

0.08126157218579984

3

Median absolute error

0.011479740269680015

0.053269025524179094

4

Mean squared error

0.0016633580971493562

0.014459893084060526

5

Sample size

2001

2001

\begin{Shaded}
\begin{Highlighting}[]
\NormalTok{test\_iter }\OperatorTok{=} \BuiltInTok{iter}\NormalTok{(my\_dl)}
\end{Highlighting}
\end{Shaded}

\begin{Shaded}
\begin{Highlighting}[]
\NormalTok{data }\OperatorTok{=} \BuiltInTok{next}\NormalTok{(test\_iter)}
\NormalTok{fake\_imgs, real\_imgs }\OperatorTok{=}\NormalTok{ visualize(model, data, }\VariableTok{False}\NormalTok{)}
\end{Highlighting}
\end{Shaded}

\begin{figure}
\centering
\includegraphics{Project_files/Project_48_0.png}
\caption{png}
\end{figure}

\begin{Shaded}
\begin{Highlighting}[]
\KeywordTok{def}\NormalTok{ write\_image(image:np.ndarray, image\_path: }\BuiltInTok{str}\NormalTok{):}
    \CommentTok{\textquotesingle{}\textquotesingle{}\textquotesingle{}}
\CommentTok{    Writes image from image path}
\CommentTok{    Args:}
\CommentTok{        image: RGB image of shape H x W x C, with float32 data}
\CommentTok{        image\_path: path to image}

\CommentTok{    Returns:}
\CommentTok{        RGB image of shape H x W x 3 in floating point format}
\CommentTok{    \textquotesingle{}\textquotesingle{}\textquotesingle{}}
    \CommentTok{\# read image and convert to RGB}
\NormalTok{    bgr\_image }\OperatorTok{=}\NormalTok{ (image[:, :, [}\DecValTok{2}\NormalTok{, }\DecValTok{1}\NormalTok{, }\DecValTok{0}\NormalTok{]] ).astype(np.uint8)}
\NormalTok{    cv2.imwrite(image\_path, bgr\_image)}
\end{Highlighting}
\end{Shaded}

\begin{Shaded}
\begin{Highlighting}[]
\KeywordTok{def}\NormalTok{ rescale\_img\_with\_colors (grayscale, pred):}
    
\NormalTok{    h,w }\OperatorTok{=}\NormalTok{ grayscale.shape}
\NormalTok{    f }\OperatorTok{=}\NormalTok{ cv2.resize(pred, (w,h))}
\NormalTok{    f }\OperatorTok{=}\NormalTok{ cv2.cvtColor(f, cv2.COLOR\_RGB2LAB)}
\NormalTok{    grayscale }\OperatorTok{=}\NormalTok{ np.}\BuiltInTok{abs}\NormalTok{(grayscale}\OperatorTok{*}\FloatTok{0.4}\NormalTok{).astype(}\StringTok{\textquotesingle{}uint8\textquotesingle{}}\NormalTok{)}
\NormalTok{    new  }\OperatorTok{=}\NormalTok{ np.stack((grayscale, f[...,}\DecValTok{1}\NormalTok{], f[...,}\DecValTok{2}\NormalTok{]), axis}\OperatorTok{={-}}\DecValTok{1}\NormalTok{)}
\NormalTok{    new }\OperatorTok{=}\NormalTok{ cv2.cvtColor(new, cv2.COLOR\_LAB2RGB) }\OperatorTok{*} \FloatTok{255.}
    \ControlFlowTok{return}\NormalTok{ new.astype(}\StringTok{\textquotesingle{}uint8\textquotesingle{}}\NormalTok{)}
    
\end{Highlighting}
\end{Shaded}

\begin{Shaded}
\begin{Highlighting}[]
\NormalTok{test\_paths }\OperatorTok{=}\NormalTok{ glob.glob(}\StringTok{"portraits/*.JPG"}\NormalTok{)}
\NormalTok{my\_dl }\OperatorTok{=}\NormalTok{ make\_dataloaders(paths}\OperatorTok{=}\NormalTok{test\_paths, split}\OperatorTok{=}\StringTok{\textquotesingle{}val\textquotesingle{}}\NormalTok{,  n\_workers}\OperatorTok{=}\DecValTok{0}\NormalTok{) }
\NormalTok{my\_iter }\OperatorTok{=} \BuiltInTok{iter}\NormalTok{(my\_dl)}
\NormalTok{i }\OperatorTok{=} \DecValTok{0}
\ControlFlowTok{for}\NormalTok{ data }\KeywordTok{in}\NormalTok{ tqdm(my\_iter):}
\NormalTok{    fake\_imgs, real\_imgs }\OperatorTok{=}\NormalTok{ model\_eval(model, data)}
    \ControlFlowTok{for}\NormalTok{ fimg }\KeywordTok{in}\NormalTok{ fake\_imgs:}
\NormalTok{        imagefn }\OperatorTok{=}\NormalTok{ test\_paths[i]}
\NormalTok{        i }\OperatorTok{+=} \DecValTok{1}
\NormalTok{        original }\OperatorTok{=}\NormalTok{ cv2.imread(imagefn, cv2.IMREAD\_GRAYSCALE)}
\NormalTok{        color }\OperatorTok{=}\NormalTok{ rescale\_img\_with\_colors(original, fimg)}
\NormalTok{        write\_image(color, }\StringTok{"results/"}\OperatorTok{+}\NormalTok{imagefn)}
\NormalTok{        fig, axes }\OperatorTok{=}\NormalTok{ plt.subplots(}\DecValTok{1}\NormalTok{, }\DecValTok{3}\NormalTok{, figsize}\OperatorTok{=}\NormalTok{(}\DecValTok{15}\NormalTok{,}\DecValTok{15}\NormalTok{))}
\NormalTok{        axes[}\DecValTok{0}\NormalTok{].imshow(original, cmap}\OperatorTok{=}\StringTok{\textquotesingle{}gray\textquotesingle{}}\NormalTok{)}
\NormalTok{        axes[}\DecValTok{1}\NormalTok{].imshow(fimg)}
\NormalTok{        axes[}\DecValTok{2}\NormalTok{].imshow(color)}
\NormalTok{        plt.show()}
\end{Highlighting}
\end{Shaded}

\begin{verbatim}
c:\users\fnoya\appdata\local\programs\python\python37\lib\site-packages\ipykernel_launcher.py:10: DeprecationWarning: BICUBIC is deprecated and will be removed in Pillow 10 (2023-07-01). Use Resampling.BICUBIC instead.
  # Remove the CWD from sys.path while we load stuff.



  0%|          | 0/1 [00:00<?, ?it/s]
\end{verbatim}

\begin{figure}
\centering
\includegraphics{Project_files/Project_51_2.png}
\caption{png}
\end{figure}

\begin{figure}
\centering
\includegraphics{Project_files/Project_51_3.png}
\caption{png}
\end{figure}

\begin{figure}
\centering
\includegraphics{Project_files/Project_51_4.png}
\caption{png}
\end{figure}

\begin{figure}
\centering
\includegraphics{Project_files/Project_51_5.png}
\caption{png}
\end{figure}

\begin{figure}
\centering
\includegraphics{Project_files/Project_51_6.png}
\caption{png}
\end{figure}

\begin{figure}
\centering
\includegraphics{Project_files/Project_51_7.png}
\caption{png}
\end{figure}

\begin{figure}
\centering
\includegraphics{Project_files/Project_51_8.png}
\caption{png}
\end{figure}

\begin{figure}
\centering
\includegraphics{Project_files/Project_51_9.png}
\caption{png}
\end{figure}

\begin{figure}
\centering
\includegraphics{Project_files/Project_51_10.png}
\caption{png}
\end{figure}

\begin{figure}
\centering
\includegraphics{Project_files/Project_51_11.png}
\caption{png}
\end{figure}

\begin{figure}
\centering
\includegraphics{Project_files/Project_51_12.png}
\caption{png}
\end{figure}

\begin{Shaded}
\begin{Highlighting}[]
\OperatorTok{\%}\NormalTok{matplotlib notebook}
\CommentTok{\#\%matplotlib widget}

\NormalTok{mask\_coords }\OperatorTok{=}\NormalTok{ utils.specify\_mask(img)}
\OperatorTok{\%}\NormalTok{matplotlib inline }
\end{Highlighting}
\end{Shaded}

\begin{verbatim}
If it doesn't get you to the drawing mode, then rerun this function again.



<IPython.core.display.Javascript object>
\end{verbatim}

\begin{Shaded}
\begin{Highlighting}[]
\NormalTok{xs }\OperatorTok{=}\NormalTok{ mask\_coords[}\DecValTok{0}\NormalTok{]}
\NormalTok{ys }\OperatorTok{=}\NormalTok{ mask\_coords[}\DecValTok{1}\NormalTok{]}
\OperatorTok{\%}\NormalTok{matplotlib inline}
\ImportTok{import}\NormalTok{ matplotlib.pyplot }\ImportTok{as}\NormalTok{ plt}
\NormalTok{plt.figure()}
\NormalTok{fg\_mask }\OperatorTok{=}\NormalTok{ utils.get\_mask(ys, xs, img)}
\end{Highlighting}
\end{Shaded}

\begin{verbatim}
---------------------------------------------------------------------------

ValueError                                Traceback (most recent call last)

<ipython-input-21-bb4a2f9fbf03> in <module>
      4 import matplotlib.pyplot as plt
      5 plt.figure()
----> 6 fg_mask = utils.get_mask(ys, xs, img)


~/SageMaker/CS445-Colorization-Project/utils.py in get_mask(ys, xs, img)
    115 
    116 def get_mask(ys, xs, img):
--> 117     mask = poly2mask(ys, xs, img.shape[:2]).astype(int)
    118     fig = plt.figure()
    119     plt.imshow(mask, cmap='gray')


~/SageMaker/CS445-Colorization-Project/utils.py in poly2mask(vertex_row_coords, vertex_col_coords, shape)
      4 
      5 def poly2mask(vertex_row_coords, vertex_col_coords, shape):
----> 6     fill_row_coords, fill_col_coords = draw.polygon(vertex_row_coords, vertex_col_coords, shape)
      7     mask = np.zeros(shape, dtype=np.bool)
      8     mask[fill_row_coords, fill_col_coords] = True


~/anaconda3/envs/mxnet_latest_p37/lib/python3.7/site-packages/skimage/draw/draw.py in polygon(r, c, shape)
    497 
    498     """
--> 499     return _polygon(r, c, shape)
    500 
    501 


skimage/draw/_draw.pyx in skimage.draw._draw._polygon()


~/anaconda3/envs/mxnet_latest_p37/lib/python3.7/site-packages/numpy/core/_methods.py in _amin(a, axis, out, keepdims, initial, where)
     32 def _amin(a, axis=None, out=None, keepdims=False,
     33           initial=_NoValue, where=True):
---> 34     return umr_minimum(a, axis, None, out, keepdims, initial, where)
     35 
     36 def _sum(a, axis=None, dtype=None, out=None, keepdims=False,


ValueError: zero-size array to reduction operation minimum which has no identity



<Figure size 640x480 with 0 Axes>
\end{verbatim}

\begin{Shaded}
\begin{Highlighting}[]
\NormalTok{fgModel }\OperatorTok{=}\NormalTok{ np.zeros((}\DecValTok{1}\NormalTok{, }\DecValTok{65}\NormalTok{), dtype}\OperatorTok{=}\StringTok{"float"}\NormalTok{)}
\NormalTok{bgModel }\OperatorTok{=}\NormalTok{ np.zeros((}\DecValTok{1}\NormalTok{, }\DecValTok{65}\NormalTok{), dtype}\OperatorTok{=}\StringTok{"float"}\NormalTok{)}
\NormalTok{fg\_mask[fg\_mask}\OperatorTok{==}\DecValTok{1}\NormalTok{] }\OperatorTok{=}\NormalTok{ cv2.GC\_PR\_FGD}
\NormalTok{img }\OperatorTok{=}\NormalTok{ np.uint8(img}\OperatorTok{*}\DecValTok{255}\NormalTok{)}
\NormalTok{fg\_mask }\OperatorTok{=}\NormalTok{ np.uint8(fg\_mask)}
\NormalTok{(mask, bgModel, fgModel) }\OperatorTok{=}\NormalTok{ cv2.grabCut(img, fg\_mask, }\VariableTok{None}\NormalTok{, bgModel, }\OperatorTok{\textbackslash{}}
\NormalTok{                                       fgModel, iterCount}\OperatorTok{=}\DecValTok{5}\NormalTok{, mode}\OperatorTok{=}\NormalTok{cv2.GC\_INIT\_WITH\_MASK)}

\NormalTok{mask }\OperatorTok{=}\NormalTok{ np.where((mask}\OperatorTok{==}\DecValTok{2}\NormalTok{)}\OperatorTok{|}\NormalTok{(mask}\OperatorTok{==}\DecValTok{0}\NormalTok{),}\DecValTok{0}\NormalTok{,}\DecValTok{1}\NormalTok{).astype(}\StringTok{\textquotesingle{}uint8\textquotesingle{}}\NormalTok{)}
\NormalTok{fg\_img }\OperatorTok{=}\NormalTok{ img}\OperatorTok{*}\NormalTok{mask[:,:,np.newaxis]}
\NormalTok{plt.imshow(fg\_img),plt.colorbar(),plt.show()}
\end{Highlighting}
\end{Shaded}

\begin{Shaded}
\begin{Highlighting}[]
\NormalTok{img\_gray }\OperatorTok{=}\NormalTok{ cv2.cvtColor(cv2.cvtColor(img, cv2.COLOR\_RGB2GRAY), cv2.COLOR\_GRAY2RGB) }
\NormalTok{mask }\OperatorTok{=}\NormalTok{ mask[:,:,np.newaxis]}
\NormalTok{final\_img }\OperatorTok{=}\NormalTok{ img\_gray }\OperatorTok{*}\NormalTok{ (}\DecValTok{1} \OperatorTok{{-}}\NormalTok{ mask) }\OperatorTok{+}\NormalTok{ fg\_img }\OperatorTok{*}\NormalTok{ mask}
\NormalTok{plt.imshow(final\_img),plt.show()}
\end{Highlighting}
\end{Shaded}

\#\# References

\begin{itemize}
\tightlist
\item
  https://pyimagesearch.com/2020/07/27/opencv-grabcut-foreground-segmentation-and-extraction/
\item
  https://docs.opencv.org/3.4/d8/d83/tutorial\_py\_grabcut.html
\item
  https://towardsdatascience.com/colorizing-black-white-images-with-u-net-and-conditional-gan-a-tutorial-81b2df111cd8
\item
  https://doi.org/10.48550/arXiv.2106.06321
\item
  https://machinelearningmastery.com/generative-adversarial-network-loss-functions/
\item
  https://arxiv.org/abs/1711.10337 Are GANs Created Equal? A Large-Scale
  Study, 2018.
\item
  https://arxiv.org/abs/1406.2661 Generative Adversarial Networks
\item
  https://arxiv.org/pdf/1704.00028.pdf Improved Training of Wasserstein
  GANs
\item
  https://arxiv.org/abs/1611.04076 Least Squares Generative Adversarial
  Networks
\end{itemize}

\hypertarget{refs}{}
\begin{CSLReferences}{1}{0}
\leavevmode\vadjust pre{\hypertarget{ref-colorizationtutorial}{}}%
\emph{Colorizing black \& white images with u-net and conditional GAN
--- a tutorial}. (n.d.).
\url{https://towardsdatascience.com/colorizing-black-white-images-with-u-net-and-conditional-gan-a-tutorial-81b2df111cd8}.

\leavevmode\vadjust pre{\hypertarget{ref-ViT}{}}%
Dosovitskiy, A., Beyer, L., Kolesnikov, A., Weissenborn, D., Zhai, X.,
Unterthiner, T., Dehghani, M., Minderer, M., Heigold, G., Gelly, S.,
Uszkoreit, J., \& Houlsby, N. (2020). An image is worth 16x16 words:
Transformers for image recognition at scale. \emph{CoRR},
\emph{abs/2010.11929}. \url{https://arxiv.org/abs/2010.11929}

\leavevmode\vadjust pre{\hypertarget{ref-fastai}{}}%
\emph{Fast.ai}. (n.d.). \url{https://www.fast.ai/}.

\leavevmode\vadjust pre{\hypertarget{ref-goodfellow2014}{}}%
Goodfellow, I. J., Pouget-Abadie, J., Mirza, M., Xu, B., Warde-Farley,
D., Ozair, S., Courville, A., \& Bengio, Y. (2014). \emph{Generative
adversarial networks}. \url{https://doi.org/10.48550/ARXIV.1406.2661}

\leavevmode\vadjust pre{\hypertarget{ref-pix2pix}{}}%
Isola, P., Zhu, J.-Y., Zhou, T., \& Efros, A. A. (2016). Image-to-image
translation with conditional adversarial networks. \emph{CoRR},
\emph{abs/1611.07004}. \url{http://arxiv.org/abs/1611.07004}

\leavevmode\vadjust pre{\hypertarget{ref-cocodataset}{}}%
Lin, T.-Y., Maire, M., Belongie, S. J., Bourdev, L. D., Girshick, R. B.,
Hays, J., Perona, P., Ramanan, D., Dollár, P., \& Zitnick, C. L. (2014).
Microsoft {COCO:} Common objects in context. \emph{CoRR},
\emph{abs/1405.0312}. \url{http://arxiv.org/abs/1405.0312}

\leavevmode\vadjust pre{\hypertarget{ref-transformers}{}}%
Vaswani, A., Shazeer, N., Parmar, N., Uszkoreit, J., Jones, L., Gomez,
A. N., Kaiser, L., \& Polosukhin, I. (2017). Attention is all you need.
\emph{CoRR}, \emph{abs/1706.03762}.
\url{http://arxiv.org/abs/1706.03762}

\leavevmode\vadjust pre{\hypertarget{ref-ZhangIE16}{}}%
Zhang, R., Isola, P., \& Efros, A. A. (2016). Colorful image
colorization. \emph{CoRR}, \emph{abs/1603.08511}.
\url{http://arxiv.org/abs/1603.08511}

\end{CSLReferences}

\end{document}
